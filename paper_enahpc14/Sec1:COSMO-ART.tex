\subsection{Model description}
\label{subsec:1.1}
COSMO-ART  (ART  stands  for   Aerosols  and  Reactive  Trace  gases),
developed  at  KIT  Karlsruhe  \citep{Vogel-2009}, is  a  regional  to
continental scale model coupled  online to the COSMO numerical weather
prediction  (NWP) and  climate  model \citep{Baldauf-2011}.   Physical
processes  like  transport,  turbulent  diffusion,  and  dry  and  wet
deposition  are  treated  together  with  photochemistry  and  aerosol
dynamics  using the  modal approach.  

Aerosols dynamics  are simulated  with the modal  aerosol microphysics
module  MADE \citep{Ackermann-1998}, expanded  in MADEsoot  to feature
explicit  treatment of  soot aging  through condensation  of inorganic
\citep{Riemer-2003} and organic substances. MADE describes the aerosol
population  through  five   modes  representing  sub-micron  particles
consisting of sulphate, ammonium, nitrate, particulate organic matter,
water and soot  \citep{Riemer-2004} in a range of  mixing state. These
modes are coupled  with the gas phase by  condensation and nucleation,
and are  strongly influenced by  anthropogenic emissions of  gases and
particles. MADEsoot  describes the sub-micrometer  aerosol population,
composed  of sea  salt \citep{Lundgren-2013},  mineral  dust particles
\citep{Vogel-2006,  Stanelle-2010},   by  means  of   six  interacting
log-normal modes.  For each  mode, mass contributions and total number
concentration are prognostic  quantities, while the standard deviation
is fixed.

Specific  modules are included  to simulate  the dispersion  of pollen
grains    \citep{Vogel-2008}   and    other    biological   particles.
Meteorologically-influenced emissions  are also online  coupled within
the  model  system.  The  biogenic  VOC  (volatile organic  compounds)
emissions are  calculated as functions of  the land use  type based on
the Global  Land Cover 2000  dataset and the modeled  temperatures and
radiative fluxes \citep{Vogel-1995}.

The gaseous chemistry in COSMO-ART  is solved by a modified version of
the  Regional  Acid  Deposition  Model, Version  2  (RADM2)  mechanism
\citep{Stockwell-1990},  which  is   extended  to  describe  secondary
organic aerosol formation based on a volatile basis set (VBS) approach
\citep{Athanasopoulou-2013}  and  hydroxyl  radical recycling  due  to
isoprene chemistry \citep{Geiger-2003}.  The thermodynamic equilibrium
between the  gas and particulate  phases of the inorganic  material is
achieved through the  ISORROPIA II module \citep{Fountoukis-2007}. 

COSMO-ART  is  fully  online-coupled,  and  allows  for  feedbacks  of
aerosols  on  temperature,  radiation  and cloud  condensation  nuclei
(CCN).   Analytical  description  of  modules  incorporated  for  this
purpose  exists in  \citep{Vogel-2009,  Bangert-2011}.  The  radiation
scheme used  within the  model to calculate  the vertical  profiles of
shortwave and longwave radiative fluxes is GRAALS \citep{Ritter-1992}.
In order to account for  the interaction of aerosol particles with the
cloud microphysics and radiation,  COSMO-ART uses the two-moment cloud
microphysics  scheme of  Seifert and  Beheng  \citep{Seifert-2006} and
comprehensive parameterizations for cloud condensation and ice crystal
nucleation  \citep{Bangert-2011, Bangert-2012}.   The system  of stiff
ordinary differential equations described by chemical reactions in the
aqueous-phase together with the transfer reactions is solved using the
Kinetic PreProcessor (KPP, \citealp{Damian-2002}).

Detailed  model  description  can   be  found  in  the  aforementioned
publications  as   well  as  in   \citep{Stanelle-2010,  Bangert-2012,
  Knote-2011, Knote-2013}.

\subsection{Model setup}
\label{subsec:1.2}
To define a baseline within a code under development, it was necessary
to  find  a  run-configuration  capable  of  being  recreated  in  all
subsequent versions. The  energy-to-solution benchmarking of COSMO-ART
concerns  one-day  long simulations  without  spin-up  stage, using  a
222x216x40  discretization of  Europe and  a  time step  of 120s.  The
baseline code incorporates 34 2-d and  45 3-d fields to be written out
every hour for a simulation starting  on April 13th, which is close to
the equinox and thus brings  benefits of having approximately half day
of night and  half day of sun exposure, to  ensure a proper activation
of  the chemistry  cycle. Ultimately  new  input files  closer to  the
equinox (on March 20th) will be  created, but the current set of input
files is fine for the time being.

This COSMO-ART  version is configured  to deal with a  semi Lagrangian
horizontal advection scheme  with tricubic interpolation and selective
filling diffusion option in combination with the Runge-Kutta dynamical
core.   Concerning the modelling  of wet  deposition in  aerosols, the
baseline  has  only  indirect  cloud  feedbacks  but  doesn't  include
in-cloud  scavenging (rainout)  and  below-cloud scavenging  (washout)
yet.  Amongst  physical parameterizations, precipitation  formation is
performed  by a  two-moment  cloud microphysics  scheme  instead of  a
classical  bulk microphysics  scheme. Another  important point  is the
fact that  this version makes  use of the Kinetic  PreProcessor solver
for  the  resolution of  atmospheric  chemistry ordinary  differential
equations.
