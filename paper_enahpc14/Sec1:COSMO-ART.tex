\subsection{Model description}
\label{subsec:1.1}

COSMO-ART  (ART   stands  for  Aerosols  and   Reactive  Trace  gases,
\citealp{Vogel-2009}),  developed  at KIT  Karlsruhe  (Germany), is  a
regional  to  continental scale  model  coupled  online  to the  COSMO
numerical    weather    prediction    (NWP)    and    climate    model
\citep{Baldauf-2011}. COSMO is used  by several European countries for
operational weather  forecast. The  gaseous chemistry in  COSMO-ART is
solved by  a modified version  of the Regional Acid  Deposition Model,
Version  2 (RADM2)  mechanism \citep{Stockwell-1990},  which  has been
extended   to    describe   secondary   organic    aerosol   formation
\citep{Schell-2001, Athanasopoulou-2013} and hydroxyl recycling due to
isoprene chemistry \citep{Geiger-2003}  and heterogeneous reactions as
hydrolysis  of N2O5.  Aerosols are  represented by  the  modal aerosol
module MADE \citep{Ackermann-1998},  improved by explicit treatment of
soot aging  through condensation of  inorganic \citep{Riemer-2003} and
organic substances (MADEsoot).

The  five modes that  represent the  aerosol population  contain: pure
soot, secondary mixtures of sulfates, nitrates, ammonium, organics and
water (nucleation  and accumulation) and the internal  mixtures of all
these species in  both modes. Separate fine and  coarse emission modes
for  sea-salt,  dust  \citep{Stanelle-2010},  and  rest  anthropogenic
species  are treated by  six additional  modes.  Specific  modules are
included    to   simulate    the   dispersion    of    pollen   grains
\citep{Vogel-2008}   and  other  biological   particles.   Meteorology
affected emissions are also online coupled within the model system.

\subsection{Model setup}
\label{subsec:1.2}
