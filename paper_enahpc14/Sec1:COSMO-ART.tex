This   study   uses   the   regional   atmospheric   model   COSMO-ART
\citep{Vogel-2009} (ART stands for Aerosols and Reactive Trace gases),
in which atmospheric chemistry is online-coupled to the COSMO regional
numerical      weather      prediction      and     climate      model
\citep{Baldauf-2011}. The online coupling  is realized in a consistent
way using the same spatial and temporal resolution for all scalars, as
for    example   advection,   the    same   numerical    schemes   are
applied. Detailed model description can be found in the aforementioned
publications as well as in \citep{Stanelle-2010}, \citep{Bangert-2012}
and \citep{Knote-2011}.

The gaseous chemistry in COSMO-ART  is simulated by a modified version
of  the Regional Acid  Deposition Model,  Version 2  (RADM2) mechanism
\citep{Stockwell-1990},  which  is   extended  to  describe  secondary
organic  aerosol formation  \citep{Schell-2001}  and hydroxyl  radical
recycling  due  to  isoprene chemistry  \citep{Geiger-2003}.   Aerosol
dynamics   are  simulated   with   the  modal   aerosol  module   MADE
\citep{Ackermann-1998}, improved  by explicit treatment  of soot aging
through  condensation  of  inorganic \citep{Riemer-2003}  and  organic
substances.   The five  modes  that represent  the aerosol  population
contain:  pure   soot,  secondary  mixtures   of  sulfates,  nitrates,
ammonium, organics  and water (nucleation and  accumulation mode), and
the  internal mixtures  of all  these species  in both  modes. Aerosol
species are transferred among  these mixtures through condensation and
coagulation.   A  separate  coarse  mode contains  additional  primary
emitted  particles.    Meteorologically-affected  emissions  are  also
online-coupled to the model  system. Sea-salt and desert dust emission
algorithms are described in \citep{Vogel-2006} and \citep{Vogel-2009}.
The biogenic VOC emissions are calculated as functions of the land use
type    based    on   the    Global    Land    Cover   2000    dataset
\citep{Bartholome-2005}  and the  modeled  temperatures and  radiative
fluxes  \citep{Vogel-1995}.   Sedimentation,  advection,  washout  and
turbulent diffusion also modify the aerosol distributions.
