\subsection{Hardware platforms}
\label{subsec:3.1}

Benchmarks presented  in this work  were performed on  three different
platforms  and can  be reproduced  within an  expected  variance, thus
allowing a  fair comparison between the baseline  and future milestone
versions of \cosmoart.

\begin{itemize}
\item \monch is a 10-rack NEC-provided clus\-ter at CSCS composed
  of 312 compute  nodes. %and utilised by members  of the Swiss Platform
  %for Advanced Scientific  Computing (PASC) project~\citep{PASC}
  Each node is comprised of two Intel Ivy Bridge EP  E5-2660 ten-core processors
  operating at  2.2 GHz, equipped with 32\,GB of DDR3 1600\,MHz  RAM and connected  via InfiniBand Mellanox  SX6036 and FDR switches. For our  experiments, 
  a full rack constituted of 52 nodes was used. \monch is  slated to stay
  in  service  without  hardware  upgrade  for  the  duration  of  the
  Exa2Green  project,  allowing   an  identical  configuration  to  be
  recreated for future assessments of the baseline.\\

\item \pilat is a dual-socket eight-core Intel Sandy
  Bridge EP based cluster at  CSCS composed of 42 compute nodes. %and used as Piz Daint pre-post  processing cluster 
  The 2 login  nodes and
  the 42 computes nodes are consolidated into 11 twin-pair Intel E5-Series DALCO
  r2264i4t 2U scalable compute modules, each containing 4 compute
  nodes based  on two Intel Xeon E5-2670 processors operating  at 2.6\,GHz equipped with 64GB of DDR3 1600\,MHz RAM and connected  via InfiniBand Mellanox SX6036 and FDR switches. 
  For our  experiments, a  full rack constituted  of 42 standard  compute nodes
  was used. \pilat  is based  on  Intel's second-generation
  processors, conventional in  HPC systems and known to  be more power
  consuming than Ivy Bridge successor.\\

\item \tinto is  a heterogeneous cluster composed of
  28 compute nodes at UJI. For our  experiments only a subset of 16 homogeneous nodes was considered. Each of them is composed dual-socket Intel Xeon E5645 hexa-core processors 
  running at 2.40\,GHz, equipped with 288\,GB  of DDR3 1333\,MHz and connected via Infiniband QDR with a Mellanox MTS3600 switch. 

  % A  full tracing  experiment is  conducted on
  % \tinto in order to  capture an overall  power profile at a much
  % finer resolution. In this sense, we also analyze the contribution of the MPI library
  % to the energy consumption and how the use of the blocking versus polling
  % message-passing policies can potentially render energy savings.
\end{itemize}

\subsection{Power measurement framework}
\label{subsec:3.3}

We present here the two different  frameworks deployed on the testbed HPC systems to
measure  performance and  power  consumption of the baseline execution of \cosmoart.

\subsubsection{E3METER metering framework}

Supercomputer  clusters considered  for our  experiments at CSCS of ETH Zurich are equipped with
E3METER Intelligent  Power Strips (IPS)  and Monitors (IPM)  which are
high   quality   electricity  meters   released   by  Riedo   Networks
\citep{Riedonetworks},   that  enable   to  monitor   and   log  power
consumption  of the IT  infrastructure as  well as  constantly analyse
line  voltage,  current, power-factor,  frequency  with 1\,\%  accuracy.
Using  reliable narrowband  powerline communication  (PLC) technology,
all  metering and  power  quality  data from  each  IPS are  centrally
collected  by the E3METER  Data Concentrator,  via the  existing power
cables thus  avoiding the need for  extra cabling.  This  data is made
available via  SNMP, HTTP, TELNET  through the built-in  Fast Ethernet
port.   Time  synchronisation  is  guaranteed by  using  NTP  servers.
Measured  data is  accessed  through the  open  source Cacti  software
including the E3METER Cacti Plugin  to scan the entire PLC network and
monitor in real-time the power usage of individual rack, recorded in 5
minutes interval periods.

\subsubsection{Power-performance tracing framework}

For our experiments at Jaume I University (UJI), we employed a version
of the integrated framework presented in~\cite{energy13} that employs the \pmlib library in
combination with Extrae and Paraver, which  are profiling/tracing and
visualisation tools, respectively.

\cosmoart  is compiled and linked with the Extrae instrumentation library,
which automatically  instruments the Fortran~90 code of the model, e.g., to record MPI calls.  Next, \cosmoart  is  run on  the  nodes,  thus dissipating  certain
amount  of power.   These  nodes are connected to power distribution units (PDUs)
APC 8653 that account for the dissipated power with a sampling frequency of 1\,Hz. The power data captured by these PDUs is collected into the tracing server. The client,  meanwhile, sends a start/stop  primitive  in  order  to  gather  captured  data  by  the PDUs.

Once  the run  is finished,  a power-performance trace file containing the power 
data received from the tracing server and the instrumentation post-process is generated.
This file is finally loaded in Paraver in order to visualise the performance trace and to correlate it with the power profile generated by \cosmoart.

\subsection{Software environment}
\label{subsec:3.2}

The \cosmoart  baseline is a  pure MPI based  Fortran~90 code
currently  running  on   distributed  multi-core  systems  only.   The
software stack on the platforms was controlled using the modules
framework which gives an easy  and flexible mechanism to access to all
of the  provided compilers, tools  and applications.  For  our initial
benchmarking,   we  opted   for   the  GNU   compiler  GCC 4.8.1   on
\monch and GCC 4.8.2 on  \pilat and \tinto) using the -O3 compiler
flag  in favor  of the  Intel compilers 14.0.1,  which delivered inferior
performance.  In  addition, we installed the following implementations of MPI: MVAPICH2 1.9 on \monch and \pilat and  OpenMPI 1.6.5 on \tinto. On the other hand, we  used HDF5 1.8.12 and NetCDF  (4.3.1) libraries for  the management  of extremely  large and
complex data collections.  All computes nodes have an operating system
based on GNU/Linux featuring ``2.6.32-358.11.1.el6'' kernel in
\monch, ``3.0.101-0.15-default'' in \pilat and ``2.6.18-238.19.1.el5'' in \tinto.

A snapshot of the code,  which includes, at least conceptually, all the
information needed  to reproduce the  energy-to-solution benchmarks of
\cosmoart, was produced and run  on a 1040 cores using 20~MPI
tasks  per node on  \monch and  on a  1344 cores  using 16~MPI
tasks per node on  \pilat.  The calculated region was mapped
to the participating processors using a 2D-partitioning strategy.  The
distribution along the $x$ and $y$ coordinates was defined by setting:
$nprocx=40$  and $nprocy=26$  for \monch  and  $nprocx=28$ and
$nprocy=24$  as  $nprocx$  is   usually  kept  bigger  than  $nprocy$.
Besides,  as this  version doesn't  make use  of the  traditional GRIB
library, we  specified $nprocio=0$  for GRIB I/O.   Hyperthreading was
not considered in this study  as it previously revealed that it always
led to higher energy-to-solution.



