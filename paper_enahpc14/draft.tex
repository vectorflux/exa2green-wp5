%%%%%%%%%%%%%%%%%%%%%%% file draft.tex %%%%%%%%%%%%%%%%%%%%%%%%%
%
% This is a general template file for the LaTeX package SVJour3
% for Springer journals.          Springer Heidelberg 2010/09/16
%
% Copy it to a new file with a new name and use it as the basis
% for your article. Delete % signs as needed.
%
% This template includes a few options for different layouts and
% content for various journals. Please consult a previous issue of
% your journal as needed.
%
%%%%%%%%%%%%%%%%%%%%%%%%%%%%%%%%%%%%%%%%%%%%%%%%%%%%%%%%%%%%%%%%%%%
% First comes an example EPS file -- just ignore it and proceed on the
% \documentclass line your LaTeX will extract the file if required
\begin{filecontents*}{example.eps}
%!PS-Adobe-3.0 EPSF-3.0
%%BoundingBox: 19 19 221 221
%%CreationDate: Mon Sep 29 1997
%%Creator: programmed by hand (JK)
%%EndComments
gsave
newpath
  20 20 moveto
  20 220 lineto
  220 220 lineto
  220 20 lineto
closepath
2 setlinewidth
gsave
  .4 setgray fill
grestore
stroke
grestore
\end{filecontents*}
%
\RequirePackage{fix-cm}
%
%\documentclass{svjour3}                     % onecolumn (standard format)
%\documentclass[smallcondensed]{svjour3}     % onecolumn (ditto)
%\documentclass[smallextended]{svjour3}      % onecolumn (second format)
\documentclass[twocolumn]{svjour3}           % twocolumn
%
\smartqed % flush right qed marks, e.g. at end of proof
%
\usepackage{graphicx}
\usepackage{hyperref}
\usepackage[round, sort, numbers, authoryear]{natbib}
%
% please place your own definitions here and don't use \def but
% \newcommand{}{}
%
% Insert the name of "your journal" with
% \journalname{myjournal}
%
\begin{document}

\title{Evaluating the performance  and energy efficiency of COSMO-ART,
  a fully online coupled model  system composed of a numerical weather
  forecast model  and a  chemical transport model.   \thanks{Grants or
    other notes  about the  article that should  go on the  front page
    should be placed here.}}
%\subtitle{Write here subtitle}

%\titlerunning{Short form of title} % if too long for running head

\author{J.Charles \and M.Dolz}

%\authorrunning{Short form of author list} % if too long for running
%head

\institute{Dr. Joseph Charles \at Swiss National Supercomputing Centre
  (CSCS)\\ CH-6900  Lugano, Switzerland \\  Tel.: +41 (0) 91  610 8216
  \\  \email{joseph.charles@cscs.ch}  \\  \and  Dr.  Manuel  Dolz  \at
  University of Hamburg \\ 20148  Hamburg, Germany \\ Tel.: +49 (0) 40
  460094-404 \\ \email{manuel.dolz@informatik.uni-hamburg.de}}

\date{Received: date / Accepted: date} % The correct dates will be
                                       % entered by the editor

\maketitle

\begin{abstract}
  In this paper we present  COSMO-ART, an extension of the operational
  weather  forecast  model  of   the  German  Weather  Service  (DWD),
  developed for  the evaluation of the interactions  of reactive gases
  and aerosol particles  with the state of atmosphere  at the regional
  scale. It  includes secondary aerosols,  directly emitted components
  like  soot,  mineral  dust,  sea  salt and  biological  material  as
  pollen. Processes such  as emissions, coagulation, condensation, dry
  deposition,  wet removal,  and sedimentation  of aerosols  are taken
  into account.   The overall performance  of this application  on HPC
  systems is  analysed by a  profiling and tracing study  to determine
  hotspots, i.e.  the  parts of the program that  require the dominant
  fraction of runtime, and to identify critical paths, i.e.  sequences
  of functions that shows dominant inclusive contributions for all its
  elements. Moreover, we describe measurement devices and energy-aware
  techniques  employed  to  evaluate   the  energy  footprint  of  the
  considered  application and  to  get detailed  insights about  power
  bottlenecks.  Our approach is to improve corresponding code sections
  to  sustain high  performance  while minimizing  energy-to-solution.
  This preliminary  study sets the basis of  broader considerations to
  tackle challenges  related to energy efficient HPC  in the framework
  of the Exa2Green project (\url{http://exa2green.eu/}).

\keywords{High performance computing  \and Energy-aware computing \and
  Numerical Weather Prediction}
% \PACS{PACS code1 \and PACS code2 \and more}
% \subclass{MSC code1 \and MSC code2 \and more}
\end{abstract}

\section{Introduction}
\label{intro}
The  interaction  between aerosols  and  clouds  presents  one of  the
biggest  uncertainties  in   present  day  climate  simulations.   The
chemistry involved is complex,  but there is a concerted international
effort to model the underlying processes through models. These models,
such as  the Aerosol  Reactive Transport (ART),  developed at  the KIT
(Karlsruhe  Institute of  Technology)  , offer  a  key opportunity  to
reduce   the  climate  uncertainty,   particularly  on   the  regional
scale. These  models can be coupled  with climate models,  such as the
regional  weather  and   climate  COSMO  (Consortium  for  Small-scale
Modelling) model  developed by the German Weather  Service (DWD).  The
COSMO model  is a weather forecast  model which has  become a standard
all  over Europe.   Beside the  federal weather  forecast  stations in
Germany  (Deutscher  Wetterdienst,  DWD),  Switzerland  (MeteoSchweiz,
MCH),  Italy (Ufficio  Generale  Spazio Aereo  e Meteorologia,  USAM),
Greece  (Hellenic   National  Meteorological  Service,   HNMS)  Poland
(Institute  of   Meteorology  and  Water   Management,  IMGW)  Romania
(National  Meteorological  Administration,  NMA) and  Russia  (Federal
Service for Hydrometeorology  and Environmental Monitoring, RHM), also
a   large  number   of  agencies   including  military   and  research
institutions  base their  forecasts on  COSMO. The  combined COSMO-ART
model is capable of simulating  aerosol distributions as well as their
interactions over Europe as well as other regional domains.

It has  been a major  scientific achievement to effectively  model the
underlying  chemical  processes   through  these  applications.   This
achievement  now   needs  to  be  fostered  by   reducing  its  energy
expense. Currently, resource allocation  is one of the key limitations
to scientists  in running the simulation periods  and resolutions they
would like.


\section{COSMO-ART model description}
\label{sec:1}
\subsection{Model description}
\label{subsec:1.1}

COSMO-ART  (ART   stands  for  Aerosols  and   Reactive  Trace  gases,
\citealp{Vogel-2009}),  developed  at KIT  Karlsruhe  (Germany), is  a
regional  to  continental scale  model  coupled  online  to the  COSMO
numerical    weather    prediction    (NWP)    and    climate    model
\citep{Baldauf-2011}. COSMO is used  by several European countries for
operational weather  forecast. The  gaseous chemistry in  COSMO-ART is
solved by  a modified version  of the Regional Acid  Deposition Model,
Version  2 (RADM2)  mechanism \citep{Stockwell-1990},  which  has been
extended   to    describe   secondary   organic    aerosol   formation
\citep{Schell-2001, Athanasopoulou-2013} and hydroxyl recycling due to
isoprene chemistry \citep{Geiger-2003}  and heterogeneous reactions as
hydrolysis  of N2O5.  Aerosols are  represented by  the  modal aerosol
module MADE \citep{Ackermann-1998},  improved by explicit treatment of
soot aging  through condensation of  inorganic \citep{Riemer-2003} and
organic substances (MADEsoot).

The  five modes that  represent the  aerosol population  contain: pure
soot, secondary mixtures of sulfates, nitrates, ammonium, organics and
water (nucleation  and accumulation) and the internal  mixtures of all
these species in  both modes. Separate fine and  coarse emission modes
for  sea-salt,  dust  \citep{Stanelle-2010},  and  rest  anthropogenic
species  are treated by  six additional  modes.  Specific  modules are
included    to   simulate    the   dispersion    of    pollen   grains
\citep{Vogel-2008}   and  other  biological   particles.   Meteorology
affected emissions are also online coupled within the model system.

\subsection{Model setup}
\label{subsec:1.2}


\subsection{Subsection title}
\label{subsec:1}
Don't forget to  give each section and subsection  a unique label (see
Sect.~\ref{sec:1}).
\paragraph{Paragraph headings} Use paragraph headings as needed.
\begin{equation}
a^2+b^2=c^2
\end{equation}

% For one-column wide figures use
\begin{figure}
% Use the relevant command to insert your figure file.
% For example, with the graphicx package use
  \includegraphics{example.eps}
% figure caption is below the figure
\caption{Please write your figure caption here}
\label{fig:1}       % Give a unique label
\end{figure}
%
% For two-column wide figures use
\begin{figure*}
% Use the relevant command to insert your figure file.
% For example, with the graphicx package use
  \includegraphics[width=0.75\textwidth]{example.eps}
% figure caption is below the figure
\caption{Please write your figure caption here}
\label{fig:2}       % Give a unique label
\end{figure*}
%
% For tables use
\begin{table}
% table caption is above the table
\caption{Please write your table caption here}
\label{tab:1}       % Give a unique label
% For LaTeX tables use
\begin{tabular}{lll}
\hline\noalign{\smallskip}
first & second & third  \\
\noalign{\smallskip}\hline\noalign{\smallskip}
number & number & number \\
number & number & number \\
\noalign{\smallskip}\hline
\end{tabular}
\end{table}

\begin{acknowledgements}
The  authors  would  like  to  thank the  High  Performance  and  High
Productivity  Computing  (\url{www.hp2c.ch})  Initiative and  the  FP7
funding this project.)
\end{acknowledgements}

%\bibliographystyle{spbasic}  % basic style, author-year citations
%\bibliographystyle{spphys}   % APS-like style for physics
%\bibliographystyle{spmpsci}  % mathematics and physical sciences
\bibliographystyle{unsrtnat}

\bibliography{draft}  % name your BibTeX data base

\end{document}

