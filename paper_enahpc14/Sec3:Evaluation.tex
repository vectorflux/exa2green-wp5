\subsection{Sofware environment}
\label{subsec:3.1}
The  software environment  on Monch  is controlled  using  the modules
framework which gives an easy  and flexible mechanism to access to all
of  the  CSCS  provided  compilers,  tools and  applications.   It  is
particularly  useful for  testing code  portability  between different
compilers  or for  changing  between different  versions  of the  same
compiler.  INT2LM  and the COSMO-ART model are  implemented in Fortran
90 for distributed memory parallel computers using the Message Passing
Interface (MPI).  For  our initial benchmarking, we opted  for the GNU
compiler (gcc/4.8.1) with the -O3  compiler flag as it generally gives
a  good  level  of optimization  and  the  code  runs faster  in  this
configuration  than  when  compiled  with  the  intel  compiler,  also
available on  Monch.  Besides, we installed  the MPICH2 implementation
of MPI (mvapich2/1.9) as well  as the commonly used HDF5 (hdf5/1.8.12)
and NetCDF  (netcdf/4.3.1) libraries in favor of  the traditional GRIB
library  for  the  management  of  extremely large  and  complex  data
collections.    Note   that   the  Linux   2.6.32-358.11.1.el6.x86\_64
operating system was used for all the Monch compute nodes.

\subsection{Run configuration}
\label{subsec:3.2}
The COSMO-ART model uses NAMELIST-input to specify runtime parameters
splitted into several groups:

\begin{itemize}
\item LMGRID: specifying the domain and the size of the grid,
\item RUNCTL: parameters for the model run,
\item TUNING: parameters for tuning physics and dynamics,
\item DYNCTL: parameters for the adiabatic model,
\item PHYCTL: parameters for the diabatic model,
\item COSMO\_ART: parameters for gases and aerosols model,
\item DIACTL: parameters for the diagnostic calculations,
\item SATCTL: controlling computation of synthetic satellite images,
\item IOCTL: controlling the I/O environment,
\item GRIBIN: controlling the grib input,
\item GRIBOUT: controlling the grib output,
\end{itemize}

\noindent
To run COSMO-ART the following input data are necessary:
\begin{itemize}
\item  Gas phase:  Anthropogenic emissions  for different  species and
  land use data for biogenic emissions and deposition,
\item Aerosol particles: Anthropogenic emissions,
\item Mineral dust: Soil specific land use data.
\end{itemize}

A snapshot of the code, which includes, at least conceptually, all the
information needed  to reproduce the  energy-to-solution benchmarks of
COSMO-ART, was  produced and run on a  full rack of 52  nodes on Monch
(monchc[029-080]),  which represents a  total of  1040 cores  using 20
tasks per node.  The calculated region was mapped to the participating
processors  using a 2D-partitioning  strategy. The  distribution along
the  x and  y coordinates  is defined  in the  namelist  INPUT\_ORG by
setting:  $nprocx=40$   and  $nprocy=26$,  as  the   total  number  of
processors has to be equal to $nprocx \times nprocy$.  Besides we take
$nprocio=0$.   Hyperthreading  is  not  considered in  this  study  as
previous attempts  of its  use revealed that  it always led  to higher
energy-to-solution.\\

Multiple production runs of COSMO-ART were performed to illustrate the
reproducibility  of   the  baseline,  and   quantify  the  significant
uncertainties in  the power measurement, as dictated  by the available
technology (see subsection~\ref{subsec:2.2}).

\subsection{Power-measurement results}
\label{subsec:3.3}
Measuring  the power  consumption on  one entire  Cray cabinet  is the
easiest way to calculate energy-to-solution for given applications.

\begin{figure*}
  \includegraphics[width=0.4\textwidth]{Figs/NRJ_benchmark_Monch.eps}
  \caption{Isola E1 Rack 2 Total Power and Isola E1 Total Power}
  \label{fig:1}
\end{figure*}

The run  was issued successfully  twice, the start time  and execution
time for both jobs are:
\begin{itemize}
\item start=12:51:03, end=13:20:46 $\Rightarrow$ T = 00:29:43 = 1783 s
\item start=14:28:40, end=14:58:02 $\Rightarrow$ T = 00:29:22 = 1762 s
\end{itemize}

The power  consumption for  both jobs is  calculated by  averaging the
corresponding power  measurements from the Excel  file.  Power results
account for the  Isola E1 Rack 2 Total Power. As  time resolution is 5
minutes  for the  output  results, the  average  power consumption  is
computed   by  considering  6   values  for   each  single   run  (see
Table~\ref{tab:1}).

\begin{table}
  \begin{center}
    \caption{}
    \label{tab:1}
    \begin{tabular}{lll}
      \hline\noalign{\smallskip}
      Time & Measured power (W) & Average power (W)  \\
      \noalign{\smallskip}\hline\noalign{\smallskip}
      12:55:00 & 1.2355833333e+04 & 1.265852278e+04 \\ 
      13:00:00 & 1.2820266667e+04 &  \\
      13:05:00 & 1.2600743333e+04 &  \\ 
      13:10:00 & 1.2811580000e+04 &  \\
      13:15:00 & 1.2609680000e+04 &  \\
      13:20:00 & 1.2753033333e+04 &  \\
      \noalign{\smallskip}\hline\noalign{\smallskip}
      14:30:00 & 1.2019266667e+04 & 1.258640833e+04 \\ 
      14:35:00 & 1.2632253333e+04 &  \\
      14:40:00 & 1.2779040000e+04 &  \\
      14:45:00 & 1.2753180000e+04 &  \\
      14:50:00 & 1.2610476667e+04 &  \\
      14:55:00 & 1.2724233333e+04 &  \\
      \noalign{\smallskip}\hline
    \end{tabular}
  \end{center}
\end{table}

The  total  energy to  solution  is  the straightforward  calculation:\\  

Energy-to-solution  =  Integral  of  power consumption  over  the  job
duration

Often  the  average  power  consumption  times  the  job  duration  is
sufficient.

Thus, the total energy consumption corresponds to: 
\begin{itemize}
\item E  = 1783 x 12658.52278  = 22570146.11674 J $\sim$  22.57 MJ per
  day of simulation
\item E  = 1762 x 12586.40833  = 22177251.47746 J $\sim$  22.18 MJ per
  day of simulation
\end{itemize}
