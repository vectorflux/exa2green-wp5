\subsection{Model Description}
\label{subsec:1.1}
COSMO-ART  (ART  stands  for   Aerosols  and  Reactive  Trace  gases),
developed  at  KIT  Karlsruhe  \citep{Vogel-2009}, is  a  regional  to
continental scale model coupled  online to the COSMO numerical weather
prediction (NWP) and climate model \citep{Baldauf-2011}. It allows the
online calculation of reactive  trace substances and their interaction
with  the atmosphere.   Physical processes  like  transport, turbulent
diffusion,  and  dry and  wet  deposition  are  treated together  with
photochemistry and aerosol dynamics using the modal approach.

Aerosols dynamics  are simulated  with the modal  aerosol microphysics
module  MADE \citep{Ackermann-1998}, expanded  in MADEsoot  to feature
explicit  treatment of  soot aging  through condensation  of inorganic
\citep{Riemer-2003} and organic substances. MADE describes the aerosol
population  through  five   modes  representing  sub-micron  particles
consisting of sulphate, ammonium, nitrate, particulate organic matter,
water and soot  \citep{Riemer-2004} in a range of  mixing state. These
modes are coupled  with the gas phase by  condensation and nucleation,
and are  strongly influenced by  anthropogenic emissions of  gases and
particles. MADEsoot  describes the sub-micrometer  aerosol population,
composed  of sea  salt \citep{Lundgren-2013},  mineral  dust particles
\citep{Vogel-2006,  Stanelle-2010},   by  means  of   six  interacting
log-normal modes.  For each  mode, mass contributions and total number
concentration are prognostic  quantities, while the standard deviation
is fixed.

Specific  modules are included  to simulate  the dispersion  of pollen
grains    \citep{Vogel-2008}   and    other    biological   particles.
Meteorologically-influenced emissions  are also online  coupled within
the  model  system.  The  biogenic  VOC  (volatile organic  compounds)
emissions are  calculated as functions of  the land use  type based on
the Global  Land Cover 2000  dataset and the modeled  temperatures and
radiative fluxes \citep{Vogel-1995}.

The gaseous chemistry in COSMO-ART  is solved by a modified version of
the  Regional  Acid  Deposition  Model, Version  2  (RADM2)  mechanism
\citep{Stockwell-1990},  which  is   extended  to  describe  secondary
organic aerosol formation based on a volatile basis set (VBS) approach
\citep{Athanasopoulou-2013}  and  hydroxyl  radical recycling  due  to
isoprene chemistry \citep{Geiger-2003}.  The thermodynamic equilibrium
between the  gas and particulate  phases of the inorganic  material is
achieved through the  ISORROPIA II module \citep{Fountoukis-2007}. 

COSMO-ART  is  fully  online-coupled,  and  allows  for  feedbacks  of
aerosols  on  temperature,  radiation  and cloud  condensation  nuclei
(CCN).   Analytical  description  of  modules  incorporated  for  this
purpose  exists in  \citep{Vogel-2009,  Bangert-2011}.  The  radiation
scheme used  within the  model to calculate  the vertical  profiles of
shortwave and longwave radiative fluxes is GRAALS \citep{Ritter-1992}.
In order to account for  the interaction of aerosol particles with the
cloud microphysics and radiation,  COSMO-ART uses the two-moment cloud
microphysics  scheme of  Seifert and  Beheng  \citep{Seifert-2006} and
comprehensive parameterizations for cloud condensation and ice crystal
nucleation  \citep{Bangert-2011, Bangert-2012}.   The system  of stiff
ordinary differential equations described by chemical reactions in the
aqueous-phase together with the transfer reactions is solved using the
Kinetic PreProcessor (KPP, \citealp{Damian-2002}).

Detailed  model  description  can   be  found  in  the  aforementioned
publications  as   well  as  in   \citep{Stanelle-2010,  Bangert-2012,
  Knote-2011, Knote-2013}.

\subsection{Model Setup}
\label{subsec:1.2}
Establishing effective energy performance benchmarking of a code under
intense development such as COSMO-ART is a challenging task because of
the  absolute necessity that  results must  be reproducible  within an
expected  variance for the  duration of  the Exa2Green  project. After
consultation  with  the  climate  community  to  properly  define  the
baseline,  it was  necessary to  find a  run-configuration  capable of
being recreated in all subsequent versions of the code.

Thus, we give here a brief  overview of the main features of the model
setup for  our benchmark's daily  runs.  Three-dimensional simulations
are performed over  Europe for April 13th 2010, which  is close to the
equinox and thus  brings benefits of having approximately  half day of
night  and half  day  of  sun exposure,  therefore  ensuring a  proper
activation of the chemistry cycle. We consider daily 24-hour forecasts
without  any  previous spin-up  simulations.   The calculation  domain
corresponds to  the CORDEX-EU-44  domain and is  covered by a  grid of
$222\times   216$    points   with   a    horizontal   resolution   of
$0.22\,^{\circ}$,  i.e.  50  km in  both directions,  and  40 vertical
layers.   Boundary  and  initial  conditions, emissions  and  external
parameters  were obtained by  the INT2LM  pre-processing interpolation
tool, which interpolate  data from a coarse-grid driving  model to the
rotated    latitude-longitude   grid    of   the    COSMO-Model,   and
INT2LM\_trcr\_ART,  which  interpolates  aerosol  or  gaseous  species
initial and boundary conditions  and consists mostly of the definition
of a large number of  tracers. In particular when using COSMO-ART, the
following input data are necessary:

\begin{itemize}
\item  Gas phase:  Anthropogenic emissions  for different  species and
  land use data for biogenic emissions and deposition,
\item Aerosol particles: Anthropogenic emissions,
\item Mineral dust: Soil specific land use data.
\end{itemize}

The meteorological initial and  bounday conditions are obtained by the
the ECMWF  global spectral model IFS  with an update  frequency of 3h.
Boundary data  for gas-phase species are taken  from IFS-MOZART output
at   6h  temporal   resolution.   NAMELIST-input   specifying  runtime
parameters    splitted   into    several   groups    are    given   in
Table~\ref{tab:1}.  The model  setup incorporates  34 2-d  and  45 3-d
fields  to  be  written  out  every  hour  and  doesn't  include  data
assimilation methods.

\begin{table}[htbf]
  \begin{center}
    \caption{}
    \label{tab:1}
    \begin{tabular}{ll}
      \hline\noalign{\smallskip} 
      \textbf{Group} & \textbf{Description} \\
      \noalign{\smallskip}\hline\noalign{\smallskip}
      LMGRID & Grid domain and size parameters \\
      RUNCTL & Model run parameters \\
      TUNING & Physics and dynamics parameters \\
      DYNCTL & Adiabatic model parameters \\
      PHYCTL & Diabatic model parameters \\
      COSMO\_ART & Gases and aerosols model parameters \\
      DIACTL & Diagnostic calculations parameters \\
      SATCTL & Synthetic satellite images parameters \\
      IOCTL & I/O environment parameters \\
      GRIBIN & GRIB input parameters \\
      GRIBOUT & GRIB output parameters \\
     \noalign{\smallskip}\hline
    \end{tabular}
  \end{center}
\end{table}

This COSMO-ART  version is configured  to deal with a  semi Lagrangian
horizontal advection scheme  with tricubic interpolation and selective
filling diffusion option in combination with the Runge-Kutta dynamical
core. It  also makes  use of the  Kinetic PreProcessor solver  for the
resolution of  atmospheric chemistry ordinary  differential equations.
Concerning the  modelling of wet deposition in  aerosols, the baseline
has  only  indirect  cloud  feedbacks  but  doesn't  include  in-cloud
scavenging  (rainout)   and  below-cloud  scavenging   (washout)  yet.
Amongst   physical  parameterizations,   precipitation   formation  is
performed  by a  two-moment  cloud microphysics  scheme  instead of  a
classical  bulk microphysics  scheme.
