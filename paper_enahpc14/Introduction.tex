Comprehensive  scientific assessments  of climate-warming  trends over
the  past 50  years have  shown  that the  state and  dynamics of  the
Earth's  climate system  have undergone  unprecedented  major changes.
The  last report  of the  Intergovernmental Panel  on  Climate Change,
released  in  September  2013  \citep{IPCC}, states  with  95  percent
confidence that the Earth  climate system is unequivocally warming, as
evidenced  by  increases  in   global  average  atmosphere  and  ocean
temperatures,  widespread melting  of  snow and  ice,  and rising  sea
levels.   Besides it  confirms the  overwhelming  scientific consensus
that  it's extremely  likely that  increasing emissions  of greenhouse
gases  from   anthropogenic  activities,  such   as  land-use  change,
transportation, burning fossil fuels, are the primary driver.

Greenhouse gases  have long been  studied because of their  ability to
trap heat  in the atmosphere  and their significant warming  effect on
surface  and  lower  atmosphere  temperatures.   Some  of  them  occur
naturally  in the  atmosphere, such  as water  vapor,  carbon dioxide,
methane,  nitrous  oxide  and  ozone,  while  others  are  exclusively
human-induced, such as fluorinated  gases with a global warming effect
up to  23000 times greater than  carbon dioxide. Some,  such as carbon
dioxode  or nitrous  oxide,  are  long-lived as  they  persist in  the
atmosphere over time scales of  a decade to thousands of years, become
homogeneously well-mixed in the troposphere, and thus have a long-term
influence on climate change.

The Earth's average surface  temperature is regulated by the transfert
of incoming  solar shortwave  absorbed radiation and  outgoing thermal
longwave infrared radiation emitted  back into Space. Many factors can
alter  Earth's radiation  balance  such  as the  type  of surface  the
sunlight encounters  (i.e.  white glacier, dark  desert soil, forests,
grasslands, ocean surfaces, ice  caps, cities), the intensity of solar
energy,  the cloud  cover, the  reflectivity of  clouds or  gases, the
absorption by various greenhouse gases or surfaces and the emission of
heat by various materials.

Numerous  studies over  the last  decades have  demonstrated  that the
enhancement  of aerosol  concentration since  industrialization  has a
profound impact  on the climate  system.  Aerosols are  tiny particles
suspended in  the air,  including sea salt,  mineral dust,  ash, soot,
sulphate,   nitrate,  organic  carbon   and  black   carbon.  Although
atmospheric aerosols are not considered a heat-trapping greenhouse gas
and  have shorter  atmospheric  lifetimes (around  10  days), they  do
affect the  Earth's radiation balance  and have short-term  warming or
cooling  influences  of the  earth-atmosphere  system.  Aerosols  that
mainly scatter solar radiation have a cooling effect, by enhancing the
total  reflected  solar  radiation  from  the  Earth,  while  strongly
absorbing  aerosols  have  a  warming  effect.   Indirectly,  enhanced
aerosol  concentrations  can boost  cloud  formation  and  lead to  an
increase in their albedo, providing a net cooling force.

