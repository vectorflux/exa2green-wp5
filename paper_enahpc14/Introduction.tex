Comprehensive  scientific assessments  of climate-warming  trends over
the  past 50  years have  shown  that the  state and  dynamics of  the
Earth's climate system have  undergone unprecedented major changes, as
evidenced  by  increases  in   global  average  atmosphere  and  ocean
temperatures,  widespread melting  of  snow and  ice,  and rising  sea
levels.  The  suspected culprit for  this change are  human-induced or
anthropogenic factors,  such as the greenhouse gas  emissions from the
burning of  fossil fuels,  which may be  trapping more  incoming solar
radiation in the earth system.

One  factor   overwhelmingly  affects  the   uncertainty  in  defining
anthropogenic  radiative  forcing:  the  effects  of  aerosols  (small
airborne liquid,  solid or mixed  particles, \citealp{Putaud-2010}) in
the  atmosphere.  Although  they  are not  considered a  heat-trapping
greenhouse   gas  and   have  shorter   atmospheric   lifetimes,  they
significantly affected the atmospheric  radiative fluxes over the last
150    years   \citep{Koch-2009,   IPCC-2013}.     Enhanced   aerosols
concentrations can impact the climate system by reflecting (e.g.  pure
sulfates  and  nitrates)  or  absorbing  (e.g.   black  carbon)  solar
radiation  and  thereby exert  a  cooling  or  warming effect  on  the
Earth-atmosphere   system,  causing   a  so-called   direct  radiative
forcing     \citep{Charlson-1991,    Haywood-2000,    Ramanathan-2001,
Liao-2005, Bangert-2012, Lundgren-2013}.

Hence to  improve our understanding of  aerosol-cloud interactions and
reduce  uncertainties  of aerosol  effects  in  climate, the  research
community is making a  concerted international effort to represent the
underlying chemical  processes through models.  These  models, such as
ART (Aerosol Reactive Transport), developed at the Karlsruhe Institute
of Technology  (KIT), offer  a key opportunity  to reduce  the climate
uncertainty,  particularly on  the  regional scale  \citep{Knote-2011,
Bangert-2011, Knote-2013}.   These models can be  coupled with climate
models, such as the  regional weather forecast model COSMO (Consortium
for  Small-scale  Modelling), jointly  developed  by  a consortium  of
European weather centers including  the German weather service DWD and
MeteoSwiss,  and used  in the  climate version  (COSMO-CLM) by  a wide
research community.  The extended  COSMO-ART model provides a detailed
description of  air pollution chemistry and aerosol  processes, and is
mainly designed to study air quality and aerosol meteorology feedbacks
on short, episodic to annual time scales.  It is capable of simulating
aerosol  distributions as well  as their  interactions over  Europe as
well as other regional domains.

COSMO-ART is  computationally much more demanding than  the COSMO core
since a  large number of additional  tracers and processes  have to be
considered.  Thus this model is currently severely-limited in terms of
applicability and expensive in terms of energy consumption.  COSMO has
recently  been  ported  to  GPUs  within the  framework  of  the  High
Performance  and  High  Productivity  Computing (HP2C)  Initiative  to
optimize    it    for     computational    and    energy    efficiency
(\url{http://www.hp2c.ch/}).    Although   these   developments   will
facilitate the  application of COSMO for  numerical weather prediction
and climate simulations,  they do little to address  the coupling with
the ART  model extension, for which significant  investments are still
required to take it to a similar level. The efficiency of ART is being
addressed        in       the        EU        Exa2Green       project
(\url{http://exa2green.eu/}). The  ultimate goal of the  project is to
deliver a  prototype code, which  provides an energy efficiency  of at
least five times of the  baseline value.  Such an implementation would
allow  the  community  to  investigate critical  questions  at  higher
resolution  and   over  longer  periods,   at  reduced  cost   to  the
environment.

In  Sec.~\ref{sec:1}, we  briefly introduce  the  regional atmospheric
model COSMO-ART \citep{Vogel-2009} that accounts for feedbacks between
chemistry,  aerosols, radiation, and  clouds and  additionally specify
its technical setup related to the investigated performance and energy
evaluation  methods.   Sec.~\ref{sec:2}  describes the  different  HPC
systems   and  associated  power   measurement  equipments   used  for
conducting the benchmarking study and presents energy-aware techniques
employed  to   determine  the  energy  footprint   of  the  considered
application.   Sec.~\ref{sec:3}  gathers  benchmark results  from  the
Exa2Green  project and  highlights  areas where  improvements will  be
necessary for the subsequent  baseline refactoring.  Some related work
will  be  discussed  in  Sec.~\ref{sec:4} Finally,  we  conclude  with
practical  guidelines to identifying  energy saving  opportunities and
give  directions for further  developments of  the modeling  system in
Sec.~\ref{concl}.

