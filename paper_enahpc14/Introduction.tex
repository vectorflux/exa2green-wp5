The  interaction  between aerosols  and  clouds  presents  one of  the
biggest  uncertainties  in   present  day  climate  simulations.   The
chemistry involved is complex,  but there is a concerted international
effort to model the underlying processes through models. These models,
such as  the Aerosol  Reactive Transport (ART),  developed at  the KIT
(Karlsruhe  Institute of  Technology)  , offer  a  key opportunity  to
reduce   the  climate  uncertainty,   particularly  on   the  regional
scale. These  models can be coupled  with climate models,  such as the
regional  weather  and   climate  COSMO  (Consortium  for  Small-scale
Modelling) model  developed by the German Weather  Service (DWD).  The
COSMO model  is a weather forecast  model which has  become a standard
all  over Europe.   Beside the  federal weather  forecast  stations in
Germany  (Deutscher  Wetterdienst,  DWD),  Switzerland  (MeteoSchweiz,
MCH),  Italy (Ufficio  Generale  Spazio Aereo  e Meteorologia,  USAM),
Greece  (Hellenic   National  Meteorological  Service,   HNMS)  Poland
(Institute  of   Meteorology  and  Water   Management,  IMGW)  Romania
(National  Meteorological  Administration,  NMA) and  Russia  (Federal
Service for Hydrometeorology  and Environmental Monitoring, RHM), also
a   large  number   of  agencies   including  military   and  research
institutions  base their  forecasts on  COSMO. The  combined COSMO-ART
model is capable of simulating  aerosol distributions as well as their
interactions over Europe as well as other regional domains.

It has  been a major  scientific achievement to effectively  model the
underlying  chemical  processes   through  these  applications.   This
achievement  now   needs  to  be  fostered  by   reducing  its  energy
expense. Currently, resource allocation  is one of the key limitations
to scientists  in running the simulation periods  and resolutions they
would like.
