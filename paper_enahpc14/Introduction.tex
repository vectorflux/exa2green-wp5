Comprehensive  scientific assessments  of climate-warming  trends over
the  past 50  years have  shown  that the  state and  dynamics of  the
Earth's climate system have  undergone unprecedented major changes, as
evidenced  by  increases  in   global  average  atmosphere  and  ocean
temperatures,  widespread melting  of  snow and  ice,  and rising  sea
levels.   The last report  of the  Intergovernmental Panel  on Climate
Change, released  in September 2013  \citep{IPCC-2013}, concluded that
most of the observed temperature  increases since the mid 20th Century
have  been caused  by  increasing concentrations  of greenhouse  gases
resulting from  anthropogenic activities  such as fossil  fuel burning
and  deforestation.  Long-lived greenhouse  gases, for  example carbon
dioxide, methane  and nitrous oxide, are chemically  stable and linger
homogeneously well-mixed and over time scales of a decade to thousands
of years in  the atmosphere; whereas short-lived gases  such as water,
suphur  dioxide  and  carbon  monoxide  which  respond  physically  or
chemically  to changes  in temperature,  act primarily  as  a feedback
mechanism.   Some   extremly  powerful  greenhouse   gases,  such  as
fluorinated gases can even have  a warming effect on the atmosphere up
to  23000  times  greater  than  carbon  dioxide  \citep{Zehner-2012}.
Fortunately  the  European  Parliament,  which  has  gained  worldwide
recognition as a  leader in climate policy, has  passed legislation in
March 2014  to phase  out fluorinated gas  emissions by  two-thirds by
2030,  towards  the  internationally  agreed goal  of  keeping  global
warming  below  2  degrees  Celsius  compared to  the  temperature  in
pre-industrial times.

Current  cumulative releases of  human-induced gas  emissions enhances
the greenhouse  effect to  the Earth's atmosphere  and thus  creates a
large imbalance between incoming solar absorbed radiation and outgoing
longwave radiation emitted back  to space, through radiative trapping.
While the  Earth's temperature  is dependent upon  the greenhouse-like
action of  the atmosphere, the  Earth's radiation balance  is strongly
influenced by several  other factors such as the  type of surface that
sunlight first  encounters.  Forests, grasslands,  ocean surfaces, ice
caps, deserts,  and cities all absorb, reflect,  and radiate radiation
differently.   Sunlight falling  on a  white glacier  surface strongly
reflects back into space, resulting  in minimal heating of the surface
and  lower atmosphere.   Sunlight falling  on  a dark  desert soil  is
strongly absorbed,  on the other hand, and  contributes to significant
heating of the surface and lower atmosphere.  Cloud cover also affects
greenhouse  warming by  both reducing  the amount  of  solar radiation
reaching the earth's  surface and by reducing the  amount of radiation
energy emitted into space.

Aerosols are  suspensions of liquid,  solid or mixed particles  in the
air  (including sea  salt,  mineral dust,  sulphate, nitrate,  organic
carbon and  black carbon),  with highly variable  chemical composition
and  size  distribution \citep{Putaud-2010}.   Although  they are  not
considered a heat-trapping greenhouse gas and have shorter atmospheric
lifetimes, they significantly  affect the atmospheric radiative fluxes
and  are acknowledged  as one  of the  most significant  and uncertain
aspects    in    anthropogenic    forcing    over   the    last    150
years \citep{Koch-2009,  IPCC-2013}.  Enhanced aerosols concentrations
can impact the  climate system by reflecting (e.g.   pure sulfates and
nitrates)  or  absorbing  (e.g.   black carbon)  solar  radiation  and
thereby  exert a  cooling or  warming effect  on  the Earth-atmosphere
system,      causing       a      so-called      direct      radiative
forcing     \citep{Charlson-1991,    Haywood-2000,    Ramanathan-2001,
Liao-2005, Bangert-2012, Lundgren-2013}.   Depending on their size and
chemical composition, they can also  act as cloud condensation and ice
nuclei, and  thereby influence  the cloud microphysical  processes and
optical   properties  (cloud  albedo   effect,  \citealp{Twomey-1977})
through the contribution to  cloud formation.  This results in changes
in      droplet      concentrations     \citep{Albrecht-1989}      and
precipitation     \citep{Rosenfeld-2000,    Khain-2008,    Khain-2009,
Pruppacher-2010,   Seifert-2012,  Tao-2012,  Lee-2012},   producing  a
so-called  negative  indirect  radiative forcing  \citep{Haywood-2000,
Lohmann-2005,  VandenHeever-2011, Rosenfeld-2013}.  Wet  scavenging of
aerosol  particles  represents  a  major  removal  mechanism  for  air
pollutants from  the atmosphere, however the  effects of anthropogenic
aerosols on  clouds and  the hydrological cycle  as well as  the cloud
lifetime    effect    are    especially    hard    to    assess    and
quantify   \citep{IPCC-2013},   and   remain   one  of   the   largest
uncertainties in climate modeling and in climate change prediction due
to  the lack  of  knowledge in  cloud feedbacks  \citep{Sherwood-2013,
Rosenfeld-2013, Lee-2013}.

Hence to  improve our understanding of  aerosol-cloud interactions and
reduce  uncertainties  of aerosol  effects  in  climate, the  research
community is making a  concerted international effort to represent the
underlying chemical  processes through models.  These  models, such as
ART (Aerosols  and Reactive Trace  gases) extension, developed  at the
Karlsruhe Institute  of Technology (KIT),  offer a key  opportunity to
reduce   the  climate  uncertainty,   particularly  on   the  regional
scale \citep{Knote-2011, Bangert-2011,  Knote-2013}.  These models can
be coupled with climate models,  such as the regional weather forecast
model COSMO (Consortium  for Small-scale Modelling), jointly developed
by  a consortium  of  European weather  centers  including the  German
weather service  DWD and MeteoSwiss,  and used in the  climate version
(COSMO-CLM)  by a  wide  research community.   The extended  COSMO-ART
model provides  a detailed description of air  pollution chemistry and
aerosol processes,  and is  mainly designed to  study air  quality and
aerosol  meteorology  feedbacks  on  short, episodic  to  annual  time
scales.  It is capable of  simulating aerosol distributions as well as
their interactions over Europe as well as other regional domains.

COSMO-ART is  computationally much more demanding than  the COSMO core
since a  large number of additional  tracers and processes  have to be
considered.  Thus this model is currently severely-limited in terms of
applicability and expensive in terms of energy consumption.  COSMO has
recently  been  ported  to  GPUs  within the  framework  of  the  High
Performance  and  High  Productivity  Computing (HP2C)  Initiative  to
optimize it  for computational and energy  efficiency.  Although these
developments will  facilitate the  application of COSMO  for numerical
weather prediction and climate  simulations, they do little to address
the  coupling with  the  ART model  extension,  for which  significant
investments  are still required  to take  it to  a similar  level. The
efficiency of ART is being  addressed in the EU Exa2Green project. The
ultimate goal  of the  project is to  deliver a prototype  code, which
provides an energy  efficiency of at least five  times of the baseline
value.   Such   an  implementation   would  allow  the   community  to
investigate critical  questions at  higher resolution and  over longer
periods, at reduced cost  to the environment.  In Sec.~\ref{sec:1}, we
introduce the regional  atmospheric model COSMO-ART \citep{Vogel-2009}
that  accounts for feedbacks  between chemistry,  aerosols, radiation,
and   clouds.   Sec.~\ref{sec:2}  and   Sec.~\ref{sec:3}  successively
describe the  hardware and measurement  devices used in  evaluation as
well  as  energy-aware techniques  employed  to  determine the  energy
footprint  of  the  considered  application.   Sec.~\ref{sec:4}  shows
benchmark  results  from  the  Exa2Green  project  and  highlight  the
critical  components  of  the  application.  Finally,  a  summary  and
directions for model optimization are given in Sec.~\ref{concl}.

