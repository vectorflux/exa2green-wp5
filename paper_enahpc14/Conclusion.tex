We  have   presented  a  methodology  for   comparing  performance  of
\cosmoart,  a  regional  weather  forecast  model  augmented  for  the
interactions of  reactive gases and aerosol  particles.  The resulting
benchmarks illustrate  that the  best time-to-solution does  not imply
the best energy-to-solution: an Intel Sandy Bridge (2.6 GHz) system has
lower time-to-solution but higher energy-to-solution than an Intel Ivy
Bridge (2.2 GHz) system, although  the two metrics are indeed strongly
correlated.   On the  other  hand, the  usage  of energy-friendly  MPI
waiting techniques analyzed  with our power-per\-for\-man\-ce framework over
a series of  runs of \cosmoart on \tinto  and power traces demonstrate
that   it  is   possible  to   reduce  both   power   dissipation  and
energy-to-solution   while  maintaining   (or  even   decreasing)  the
time-to-solution. The resulting profiles indicate that simple changes,
such as making use of the  \emph{degraded} MPI policy rather than the \emph{aggressive}
is  possible to  slightly  reduce both  power  dissipation and  energy
consumption (by 2\,\%).

This  reproducible  benchmark provides  a  baseline  for ongoing  work
package within  the EU-funded Exa2Green to  minimize power consumption
of \cosmoart. Profiling  has given us insight into  the most expensive
code  components,  which are  now  being  altered  to utilize  revised
algorithms.  The  results of these  optimizations will be  reported in
future publications.
