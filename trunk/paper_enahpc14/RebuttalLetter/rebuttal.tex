\documentclass[a4paper,11pt]{article}

\title{Authors' Reply to Reviewers' Comments on\\ \itshape{``Evaluating the Performance and Energy Efficiency of the COSMO-ART Model System''}}

\author{Joseph~Charles, William~Sawyer, Manuel~F.~Dolz, Sandra~Catal\'an}

\usepackage[left=2.5cm,right=2.2cm,top=3cm,bottom=3cm]{geometry}
\usepackage[latin1]{inputenc}
\usepackage{color}
\usepackage{graphicx}
\usepackage{url}
\usepackage[urlcolor=blue,colorlinks=true]{hyperref}
\usepackage{ifthen}

%\newcommand{\xavi}[1]{\textcolor{blue}{#1}}
%\newcommand{\xavi}[1]{{#1}}
%\newcommand{\xaviNote}[1]{\footnote{\textcolor{blue}{#1}}}

\newcounter{question}
\setcounter{question}{1}

%\newcommand{\reviewerSay}{\vspace{0.2cm} {\bf \noindent \arabic{question}} \fbox{{\sc \textcolor{red}{Reviewer:}}} \medskip \rm}
\newcommand{\reviewerSay}{\vspace{0.2cm} {\bf \noindent } {\bf Reviewer:} \medskip \par}
\newcommand{\editorialSay}{\vspace{0.2cm} {\bf \noindent \arabic{question}.} {\bf Editorial:} \medskip \par}
%\newcommand{\weReply}{\vspace{0.25cm} {\bf \noindent \arabic{question}} \addtocounter{question}{1} \fbox{{\sc \textcolor{blue}{Authors:}}} \medskip \em\par}
\newcommand{\weReply}{\vspace{0.25cm} {\bf \noindent } \addtocounter{question}{1} \textbf{\emph{Authors:}} \medskip \em\par}
\newenvironment{weSay}{\bigskip \weReply \begin{it}}{\end{it}\begin{center} \samepage$\star \star \star$ \end{center}}
\newenvironment{weSayToEditorial}{\bigskip \weReply \begin{it}}{\end{it}\begin{center} \samepage$\star \star \star$ \end{center}}

%\newcommand{\reviewerSay}{\paragraph*{(\arabic{question}) Reviewer: }}
%\newcommand{\authorsReply}{\paragraph*{\emph{(\arabic{question}) \addtocounter{question}{1} Authors: }}}
%\newenvironment{weSay}{\vspace{-3ex} \authorsReply \begin{it}}{\end{it}\begin{center} \samepage$\star \star \star$ \end{center}}

\begin{document}

\maketitle

\noindent 
We would like to thank the Anonymous Reviewers
for providing us with a prompt and insightful review. 
We take this as an indication that there was genuine interest in the paper,
which is always a compliment regardless of the outcome of the review.

We addressed all the suggestions and recommendations from the reviewers to improve
the quality and flow of the paper. Below, we discuss the comments and suggestions
from the reviewers in detail.



\newpage
% ------------------------------------------------------------------------------------------------------------------------------------------------
\section*{Reviewer \#1}

\reviewerSay
The authors perform an energy performance and efficiency analysis of the COSMO-ART model system,
a complex weather prediction and climate model, focused on the study of aerosol concentrations in the
atmosphere. They present the model, the software and the hardware setup, including two metering
frameworks and perform out-of-band (i.e. measurement from external tools) power draw measurement
for three different setups. They use the experiments to perform two different analyses: energy 
efficiency for Sandy Bridge versus Ivy Bridge, and a power profile to compare polling and blocking 
MPI policies that also show the heavy energy consuming phases of the COSMO-ART model system.

The paper is well written and is easy to follow, and, in spite of the idea not being completely new, 
it offers a good insight about where are the most energy demanding phases and how the different 
policies on MPI affect them for the second experimentation part. The first experimentation part, 
besides being a common study, gives an interesting point of view about how energy efficient does 
not necessarily means time efficient.

%%% *** Manuel/Sandra ***
\begin{weSay}
The authors thank the reviewer for his/her general view of the paper. We consider this work as
a first step towards an energy-efficient version of COSMO-ART inside the Exa2Green project.
\end{weSay}

Some suggestions regarding the paper:

- MONCH and PILATUS are presented as Ivy Bridge and Sandy Bridge clusters respectively in Section 4.1, 
and afterwards in section 5, they are referred as Ivy Bridge and Xeon E5 respectively. Apart from not 
being consistent, both processor families are Xeon E5 thus confusing a bit on what cluster is 
being referred. This also affects Table 1.

%%% *** Joseph ***
\begin{weSay}
We have fixed this issue along the manuscript. To do so we have explained in detail the specifications of the machines
in the Environment setup section and afterwards we refer to the platforms' name or using the
full processor's name and its microarchitecture.
\end{weSay}

- It would be easier to compare values in Figure 4 if both energy ETSs and TTSs where together. 
Right now you show ETS TTS ETS TTS, and ETS ETS TTS TTS would be easier to understand.

%%% *** Joseph ***
\begin{weSay}
We have swapped the corresponding columns in the Figure 4, now ETS and TTS appear grouped.
\end{weSay}

- In multiple places: "power dissipation", "power draw". I assumed it this way, if it is not 
I suggest to explain it better.

%%% *** Manuel ***
\begin{weSay}
We have accordingly replaced ``power dissipation'' by ``power consumption'' along the manuscript.
\end{weSay}


\newpage
\section*{Reviewer \#2}

\reviewerSay
The paper investigates the energy footprint of a weather forecasting model on various HPC platforms. 
As such the paper provides the overall energy footprint covering the overall run. In other words the 
granularity is very coarse and is at the application level. Although this will be interesting to the 
COSMO-ART community, the results include some good details on traces  on the
evaluation of polling vs blocking MPI strategies.

%%% *** Manuel/Sandra ***
\begin{weSay}
The authors thank the reviewer for his/her general view of the paper.
\end{weSay}

It could have been very beneficial to describe the nature of the application in detail instead of 
citing them. I assume that authors have resorted to that technique for the reasons of brevity  - yet 
I beleive that such a detailed description is essential to understand the behaviour of the application.

%%% *** Joseph ***
\begin{weSay}
blabla
\end{weSay}

With missing crucial details of the application and in the abseence of any results towards various 
problem sizes or domains of cases, it is very difficult to do any post-analysis on the overall 
behaviour of the application - for example its runtime scalability (both weak and strong) vs energy 
scalability etc.

%%% *** Will ***
\begin{weSay}
blabla
\end{weSay}

In summary, although the paper is very interesting, the lack or depth of information in there severely 
limits the re-usable knowledge contributed by the paper.

%%% *** Will ***
\begin{weSay}
blabla
\end{weSay}


\newpage
\section*{Reviewer \#3}

The authors describe the extended atmospheric model COSMO-ART. They present
performance and energy measurements on a selection of HPC systems. There
is no new methodology presented in this paper.

%%% *** Will ***
\begin{weSay}
blabla
\end{weSay}

Related work is weak. I don't see how [20] and [11] are related to the content
of this paper (other algorithms, other architectures, no comparison of the
energy measurement infrastructure). I suggest to focus more on infrastructure
and tool support for system level power measurement in HPC environments in
the related work section. This would give you the opportunity to argue that
the 1 Hz resolution in one of your test systems is actually state-of-the-art.

%%% *** Will/Joseph ***
\begin{weSay}
blabla
\end{weSay}

The hardware description is a bit confusing.
- why does the NEC system have 10 racks for 312 nodes if one rack houses 52
     nodes?

%%% *** Joseph ***
\begin{weSay}
We have explained in detail the kind of nodes that this platform is comprised of.
We only used a subset of 52 in our experiments.
\end{weSay}

- the "v2" is missing from the Ivy Bridge processor's name string

%%% *** Joseph ***
\begin{weSay}
Fixed.
\end{weSay}

- sentence structure: "Each node is comprises two..."

%%% *** Joseph ***
\begin{weSay}
Fixed.
\end{weSay}

- what do you mean by "second-generation processor" regarding Sandy Bridge?
     SB is sometimes referred to as 2nd gen Core-i* processor but this does not
     apply to the Xeon line

%%% *** Joseph ***
\begin{weSay}
We recognize the mistake and fix the sentence in the manuscript.
\end{weSay}

- the claim that SB is known to consume more power is wrong. The TDP classes
     are the same for IB and SB (e.g. E5-2670 and E5-2670v2 are both 115W parts)
     It presumably is less efficient, but even such a claim would need a reference

%%% *** Joseph ***
\begin{weSay}
blabla
\end{weSay}

- It should be added if the Turbo Boost feature is enabled or disabled, this
     seems to be important for the discussion of the TTS results.

%%% *** Manuel/Sandra ***
\begin{weSay}
We added information about the Turbo Boost feature for each platform \textbf{(MISSING FOR MONCH and PILATUS!!)}
\end{weSay}

- It should be added which power saving mechanisms are used (cpufreq
     governor, available c-states)

%%% *** Manuel/Sandra ***
\begin{weSay}
We added information about the power saving mechanisms available on each platform \textbf{(MISSING FOR MONCH and PILATUS!!)}
\end{weSay}

- the second system is described as 16 cores x 42 nodes = 672 cores total. In
     4.3 it is stated that 1344 cores were used (Could be HyperThreading, however
     it is also stated that HT was not considered. Also, 16 MPI ranks per node
     indicate 84 nodes)

%%% *** Joseph ***
\begin{weSay}
We have fixed this mistake. HyperThreading was not considered in our experiments.
\end{weSay}

In 4.2.1
A 5 minute sampling interval seems useless to me. It would be nice to know if
this is actually an instantaneous value every 5 minutes (as I read it) or the
correct average of the past 5 minutes. Figure 5 suggests variation of the power
consumption in the 10\% range. Thus instantaneous values would have a high margin
for error and the 4,6, or 9 samples you use to calculate the average would be
too few to estimate a reliable average.

%%% *** Will/Joseph ***
\begin{weSay}
\textbf{MISSING ANSWER!!}

Figure 5 represents the power drawn of Tintorrum, for this case the PDU used has a sampling rate of 1 Hz instead of 5 minutes.
\end{weSay}

In 4.2.2
There is no information about the accuracy

%%% *** Manuel/Sandra ***
\begin{weSay}
We have added information about the accuracy of the PDU.
\end{weSay}

In 4.3
Only -O3 is not appropriate for the Sandy bridge and Ivy Bridge system. At
least -mavx should be added. It looks like some low hanging fruit is reserved
to reach the targeted 5x improvement

%%% *** Will ***
\begin{weSay}
blabla
\end{weSay}

In 5.1
The blocking mode does not seem to be blocking. As there are no other
processes/threads available (as many MPI ranks as cores per node) the waiting
ranks will repeatedly call sched\_yield() only to be picked again by the OS
scheduler. This also explains the small benefit as no idle states are used.
You should check the CPU utilization. I guess it is 100\% on all cores in both
modes. If it is not, you should check with powertop which idle states are used.

%%% *** Manuel/Sandra ***
\begin{weSay}
We noticed the problem and fixed accordingly along the manuscript the misused blocking term.
Now we refer to ``aggressive'' and ``degraded'' OpenMPI policies instead of ``polling''
and ``blocking''. The reviewer is completely right on this issue, however even when using
the ``degraded'' policy the CPU utilization is readuced to 50\,\%, thus saving energy.
\end{weSay}

In 5.2
What do you mean by "it aims at computing speed regardless to energy consumption"?
Does it mean it uses Turbo? Please clarify what exactly is different than on the
IB system.

%%% *** Joseph ***
\begin{weSay}
We have fixed the sentence accordingly so as not to be confusing.
\end{weSay}

It is confusing to compare "Ivy Bridge EP" against "Xeon E5" in the result
discussion, as the Ivy Bridge processor is also called Xeon E5 (v2) as well.

%%% *** Joseph ***
\begin{weSay}
We have fixed this issue along the manuscript.
\end{weSay}

Since the SB and IB measurements use different amounts of cores it is essential to
know if the application scales linearly with the core count. If the application
is scalable 1020 cores at 2.2 GHz should be faster than 672 cores at 2.6 GHz.

%%% *** Will ***
\begin{weSay}
blabla
\end{weSay}

In 5.3
Only 5\% energy reduction when blocked shows that the processors are not idle.
As mentioned earlier, you should add information if power management is
actually used while the processors are waiting.

%%% *** Manuel/Sandra ***
\begin{weSay}
The reviewer is right in this sense. The processors are not idle during blocking MPI routines, however a reduction of the CPU utilization is observed (50\,\%).
\end{weSay}

\end{document}
