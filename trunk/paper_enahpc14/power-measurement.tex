\subsection{Power-Performance measurement framework}

To assess the performance and the energy efficiency of COSMO-ART,
we employ a version of the integrated framework presented in~\cite{energy13}
that works in combination with VampirTrace and Vampir, which are profiling/tracing
and visualization tools, respectively. 
%The left part of the \vref{fig:Lustre} offers a graphical representation of
%the Lustre architecture; the right depicts the tracing and profiling framework.
To use our approach, COSMO-ART is compiled using the VampirTrace compiler wrappers, 
which automatically instrument the Fortran code of the model. Next, COSMO-ART 
is run on the nodes, thus dissipating certain amount of power.
The server nodes are connected to power measurement devices that account for
the dissipated power/consumed energy and send the power data to the tracing server.
The attached VampirTrace \pmlib plugin employs the client API that
sends start/stop primitives in order to gather captured data by the wattmeters onto
the tracing server, where an instance of the \pmlib server is running.
Once COSMO-ART run is finished, the VampirTrace \pmlib plugin
receives the power data from the tracing server.
The instrumentation post-process generates the performance trace files and the
\pmlib plugin inserts the power data into them.

In addition to the power measurements, we also account for the resource utilization
values of the nodes: CPU load, memory usage and storage device utilization. % and network utilization.
We run special \pmlib server instances on the server nodes
that retrieve these values from the \texttt{proc} file system (leveraging the \texttt{psutil} Python library).
Thus, \pmlib plugin instances running with the instrumented application
connect with the \pmlib servers. Finally, using the Vampir visualization tool, the power-performance traces
can be easily analyzed through a series of plots and statistics.
