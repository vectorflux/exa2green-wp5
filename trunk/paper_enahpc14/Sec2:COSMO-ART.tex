\subsection{Model Description}
\label{subsec:1.1}
COSMO-ART  (ART  stands  for   Aerosols  and  Reactive  Trace  gases),
developed  at  KIT  Karlsruhe  \citep{Vogel-2009}, is  a  regional  to
continental scale model coupled  online to the COSMO numerical weather
prediction (NWP) and climate model \citep{Baldauf-2011}. It allows the
online calculation of reactive  trace substances and their interaction
with  the atmosphere.   Physical processes  like  transport, turbulent
diffusion,  and  dry and  wet  deposition  are  treated together  with
photochemistry and aerosol dynamics using the modal approach. Detailed
model description  can be found in the  aforementioned publications as
well    as   in   \citep{Stanelle-2010,    Bangert-2012,   Knote-2011,
  Knote-2013}.

\subsection{Model Setup}
\label{subsec:1.2}
Establishing effective energy performance benchmarking of a code under
intense development such as COSMO-ART is a challenging task because of
the  absolute necessity that  results must  be reproducible  within an
expected  variance for the  duration of  the Exa2Green  project. After
consultation  with  the  climate  community  to  properly  define  the
baseline,  it was  necessary to  find a  run-configuration  capable of
being recreated in all subsequent versions of the code.

Thus, we give here a brief  overview of the main features of the model
setup for  our benchmark's daily  runs.  Three-dimensional simulations
are performed over  Europe for April 13th 2010, which  is close to the
equinox and thus  brings benefits of having approximately  half day of
night  and half  day  of  sun exposure,  therefore  ensuring a  proper
activation of the chemistry cycle. We consider daily 24-hour forecasts
without  any  previous spin-up  simulations.   The calculation  domain
corresponds to  the CORDEX-EU-44  domain and is  covered by a  grid of
$222\times   216$    points   with   a    horizontal   resolution   of
$0.22\,^{\circ}$,  i.e.  50  km in  both directions,  and  40 vertical
layers.  COSMO-ART requires the following input data:

\begin{itemize}
\item  Gas phase:  Anthropogenic emissions  for different  species and
  land use data for biogenic emissions and deposition,
\item Aerosol particles: Anthropogenic emissions,
\item Mineral dust: Soil specific land use data.
\end{itemize}

The meteorological initial and  bounday conditions are obtained by the
the ECMWF  global spectral model IFS  with an update  frequency of 3h.
Boundary data  for gas-phase species are taken  from IFS-MOZART output
at   6h  temporal   resolution.   NAMELIST-input   specifying  runtime
parameters    splitted   into    several   groups    are    given   in
Table~\ref{tab:1}.   The model setup  incorporates 34  2-d and  45 3-d
fields  to  be  written  out  every  hour  and  doesn't  include  data
assimilation methods.

\begin{table}[htbf]
  \begin{center}
    \caption{}
    \label{tab:1}
    \begin{tabular}{ll}
      \hline\noalign{\smallskip} 
      \textbf{Group} & \textbf{Description} \\
      \noalign{\smallskip}\hline\noalign{\smallskip}
      LMGRID & Grid domain and size parameters \\
      RUNCTL & Model run parameters \\
      TUNING & Physics and dynamics parameters \\
      DYNCTL & Adiabatic model parameters \\
      PHYCTL & Diabatic model parameters \\
      COSMO\_ART & Gases and aerosols model parameters \\
      DIACTL & Diagnostic calculations parameters \\
      SATCTL & Synthetic satellite images parameters \\
      IOCTL & I/O environment parameters \\
      GRIBIN & GRIB input parameters \\
      GRIBOUT & GRIB output parameters \\
     \noalign{\smallskip}\hline
    \end{tabular}
  \end{center}
\end{table}

This COSMO-ART version is configured with a semi Lagrangian horizontal
advection  scheme with  tricubic interpolation  and  selective filling
diffusion  option  in  combination   with  the  dynamical  core  using
Runge-Kutta time stepping \citep{}.  It  also makes use of the Kinetic
PreProcessor  solver  \citep{}   for  the  resolution  of  atmospheric
chemistry ordinary  differential equations.  Concerning  the modelling
of wet  deposition in aerosols,  the baseline has only  indirect cloud
feedbacks  but  doesn't  include  in-cloud  scavenging  (rainout)  and
below-cloud    scavenging    (washout)    yet.     Amongst    physical
parameterisations,   precipitation  formation   is   performed  by   a
two-moment  cloud  microphysics scheme  instead  of  a classical  bulk
microphysics scheme.
