\subsection{Model Description}
\label{subsec:1.1}

The model system \cosmoart (ART stands for Aerosols and Reactive Trace
gases) developed at KIT  Karlsruhe \cite{Vogel-2009}, is a regional to
continental scale model coupled online to the \textsc{Cosmo} numerical
weather  prediction (NWP) and  climate model  \cite{Baldauf-2011}.  It
allows the  online calculation of reactive trace  substances and their
interaction with  the atmosphere.  Physical  processes like transport,
turbulent diffusion,  and dry and wet deposition  are treated together
with photochemistry and aerosol dynamics using the modal approach.

Aerosols dynamics  are simulated  with the modal  aerosol microphysics
module  MADE \cite{Ackermann-1998},  expanded in  MADEsoot  to feature
explicit  treatment of  soot aging  through condensation  of inorganic
\cite{Riemer-2003} and organic  substances. MADE describes the aerosol
population  through  five   modes  representing  sub-micron  particles
consisting of sulphate, ammonium, nitrate, particulate organic matter,
water and  soot \cite{Riemer-2004} in  a range of mixing  state. These
modes are coupled  with the gas phase by  condensation and nucleation,
and are  strongly influenced by  anthropogenic emissions of  gases and
particles. MADEsoot  describes the sub-micrometer  aerosol population,
composed  of  sea salt  \cite{Lundgren-2013},  mineral dust  particles
\cite{Vogel-2006,   Stanelle-2010},  by   means  of   six  interacting
log-normal modes.  For each  mode, mass contributions and total number
concentration are prognostic  quantities, while the standard deviation
is fixed.

Specific  modules are included  to simulate  the dispersion  of pollen
grains    \cite{Vogel-2008}    and    other   biological    particles.
Meteorologically-influenced emissions  are also online  coupled within
the  model  system.  The  biogenic  VOC  (volatile organic  compounds)
emissions are  calculated as functions of  the land use  type based on
the Global  Land Cover 2000  dataset and the modeled  temperatures and
radiative fluxes \cite{Vogel-1995}.

The gaseous chemistry in COSMO-ART  is solved by a modified version of
the  Regional  Acid  Deposition  Model, Version  2  (RADM2)  mechanism
\cite{Stockwell-1990}, which is extended to describe secondary organic
aerosol  formation  based  on  a  volatile basis  set  (VBS)  approach
\cite{Athanasopoulou-2013}  and  hydroxyl  radical  recycling  due  to
isoprene chemistry  \cite{Geiger-2003}.  The thermodynamic equilibrium
between the  gas and particulate  phases of the inorganic  material is
achieved through the ISORROPIA II module \cite{Fountoukis-2007}.

COSMO-ART  is  fully  online-coupled,  and  allows  for  feedbacks  of
aerosols  on  temperature,  radiation  and cloud  condensation  nuclei
(CCN).   Analytical  description  of  modules  incorporated  for  this
purpose  exists  in  \cite{Vogel-2009, Bangert-2011}.   The  radiation
scheme used  within the  model to calculate  the vertical  profiles of
shortwave and longwave  radiative fluxes is GRAALS \cite{Ritter-1992}.
In order to account for  the interaction of aerosol particles with the
cloud microphysics and radiation,  COSMO-ART uses the two-moment cloud
microphysics  scheme  of Seifert  and  Beheng \cite{Seifert-2006}  and
comprehensive parameterizations for cloud condensation and ice crystal
nucleation  \cite{Bangert-2011, Bangert-2012}.   The  system of  stiff
ordinary differential equations described by chemical reactions in the
aqueous-phase together with the transfer reactions is solved using the
Kinetic PreProcessor (KPP, \cite{Damian-2002}). Further model description details  can be found in \cite{Bangert-2012,
  Knote-2011, Knote-2013}.

\subsection{Model Setup}
\label{subsec:1.2}

Establishing effective energy performance benchmarking of a code under
intense development such as \cosmoart is a challenging task because of
the  absolute necessity that  results must  be reproducible  within an
expected variance  for the duration of the  Exa2Green pro\-je\-ct.  To
define  a  baseline, it  was  necessary  to  find a  run-configuration
capable of being recreated in all subsequent versions.  Here we give a
brief  overview of  the model  setup  for our  performance and  energy
efficiency evaluation.

Three-dimensional simulations are performed over large parts of Europe
and the Mediterranean~Sea for  April 13$^{th}$ 2010, near the equinox,
in  order  to  approximately have  a  half  day  of sun  exposure  and
therefore  ensure a typical  activation of  the chemistry  cycle.  The
calculation  domain  corresponds to  the  CORDEX-EU-44  domain and  is
covered  by  a  grid  of  $222\times 216$  points  with  a  horizontal
resolution  of $0.22\,^{\circ}$,  i.e., 50~km  in both  directions and
40~vertical  layers.   These  24-hour  forecast  simulations  are  not
preceded  by a spin-up  phase to  build up  the simulated  gaseous and
aerosol   concentrations.   This  condition   means  that   the  model
initialization needs  a period of adjustment to  reach its equilibrium
state  and minimize  the effect  of initial  conditions for  gases and
aerosols on the model predictions.

The meteorological initial and boundary conditions are obtained by the
the ECMWF  global spectral model IFS  with an update  frequency of 3h.
Boundary data  for gas-phase species are taken  from IFS-MOZART output
at  6h temporal resolution.   The model  setup incorporates  34~2D and
45~3D  fields to  be written  out  every hour  and is  devoid of  data
assimilation methods.

The   considered  \cosmoart   model  system   is  configured   with  a
semi-Lagrangian   horizontal    advection   sche\-me   with   tricubic
interpolation  and selective filling  diffusion option  in combination
with   the   dynamical    core   using   Runge-Kutta   time   stepping
\cite{COSMO-PartI-2011}.    It   also  makes   use   of  the   Kinetic
PreProcessor solver  (KPP) for  the solution of  atmospheric chemistry
ordinary  differential equations  \cite{Damian-2002}.   Concerning the
modeling  of  wet  deposition  in  aerosols, the  baseline  has  only
indirect  cloud feedbacks  but  does not  include in-cloud  scavenging
(rainout) and below-cloud  scavenging (washout) yet.  Amongst physical
parameterizations,   precipitation  formation   is   performed  by   a
two-moment cloud microphysics.
