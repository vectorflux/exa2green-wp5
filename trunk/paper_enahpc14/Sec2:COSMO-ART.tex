\subsection{Model Description}
\label{subsec:1.1}
The  model  system COSMO-ART  described  in  \citep{Vogel-2009}, is  a
regional  to  continental scale  model  coupled  online  to the  COSMO
numerical weather  prediction and climate  model \citep{Baldauf-2011}.
It  incorporates  sophisticated  modules  for  gaseous  chemistry  and
aerosol dynamics  and allows the online calculation  of reactive trace
substances and their interaction  with the atmosphere.  Detailed model
description   can  be   found   in  \citep{Bangert-2012,   Knote-2011,
  Knote-2013}.

\subsection{Model Setup}
\label{subsec:1.2}
Establishing effective energy performance benchmarking of a code under
intense development such as COSMO-ART is a challenging task because of
the  absolute necessity that  results must  be reproducible  within an
expected variance for the duration of the Exa2Green project. To define
a baseline,  it was necessary  to find a run-configuration  capable of
being  recreated in  all subsequent  versions.  Here  we give  a brief
overview of the model setup  for our performance and energy efficiency
evaluation.

Three-dimensional simulations are performed over large parts of Europe
and the Mediterranean~Sea for April 13th 2010 in order to benefit from
around  half  day  of  sun  exposure and  therefore  ensure  a  proper
activation of the chemistry cycle.  The calculation domain corresponds
to the CORDEX-EU-44 domain and is covered by a grid of $222\times 216$
points with  a horizontal resolution of  $0.22\,^{\circ}$, i.e., 50~km
in  both directions  and 40~vertical  layers.  These  24-hour forecast
simulations  are not  preceded  by  a spin-up  phase  to build-up  the
simulated gaseous and  aerosol concentrations.  This condition entails
that at the  model initialization there is a  period of adjustment for
the model  to reach its equilibrium  state and minimize  the effect of
initial conditions for gases and aerosols on the model predictions.

The meteorological initial and boundary conditions are obtained by the
the ECMWF  global spectral model IFS  with an update  frequency of 3h.
Boundary data  for gas-phase species are taken  from IFS-MOZART output
at  6h temporal resolution.  The model  setup incorporates  34~2-D and
45~3-D  fields to  be written  out every  hour and  is devoid  of data
assimilation methods.

The   considered  COSMO-ART   model  system   is  configured   with  a
semi-Lagrangian    horizontal   advection    scheme    with   tricubic
interpolation  and selective filling  diffusion option  in combination
with   the   dynamical    core   using   Runge-Kutta   time   stepping
\citep{COSMO-PartI-2011}.    It  also   makes  use   of   the  Kinetic
PreProcessor solver (KPP) for  the resolution of atmospheric chemistry
ordinary  differential equations \citep{Damian-2002}.   Concerning the
modelling  of  wet  deposition  in  aerosols, the  baseline  has  only
indirect  cloud feedbacks  but  does not  include in-cloud  scavenging
(rainout) and below-cloud  scavenging (washout) yet.  Amongst physical
parameterisations,   precipitation  formation   is   performed  by   a
two-moment cloud microphysics.
