%%%%%%%%%%%%%%%%%%%%%%% file draft.tex %%%%%%%%%%%%%%%%%%%%%%%%%
%
% This is a general template file for the LaTeX package SVJour3
% for Springer journals.          Springer Heidelberg 2010/09/16
%
% Copy it to a new file with a new name and use it as the basis
% for your article. Delete % signs as needed.
%
% This template includes a few options for different layouts and
% content for various journals. Please consult a previous issue of
% your journal as needed.
%
%%%%%%%%%%%%%%%%%%%%%%%%%%%%%%%%%%%%%%%%%%%%%%%%%%%%%%%%%%%%%%%%%%%
% First comes an example EPS file -- just ignore it and proceed on the
% \documentclass line your LaTeX will extract the file if required
\begin{filecontents*}{example.eps}
%!PS-Adobe-3.0 EPSF-3.0
%%BoundingBox: 19 19 221 221
%%CreationDate: Mon Sep 29 1997
%%Creator: programmed by hand (JK)
%%EndComments
gsave
newpath
  20 20 moveto
  20 220 lineto
  220 220 lineto
  220 20 lineto
closepath
2 setlinewidth
gsave
  .4 setgray fill
grestore
stroke
grestore
\end{filecontents*}
%
\RequirePackage{fix-cm}
%
%\documentclass{svjour3}                     % onecolumn (standard format)
%\documentclass[smallcondensed]{svjour3}     % onecolumn (ditto)
%\documentclass[smallextended]{svjour3}      % onecolumn (second format)
\documentclass[twocolumn]{svjour3}           % twocolumn
%
\smartqed % flush right qed marks, e.g. at end of proof
%
\usepackage{graphicx}
\usepackage{hyperref}
\usepackage[round, sort, numbers, authoryear]{natbib}
%
% please place your own definitions here and don't use \def but
% \newcommand{}{}
%
% Insert the name of "your journal" with
% \journalname{myjournal}
%
\begin{document}

\title{Evaluating the performance  and energy efficiency of COSMO-ART,
  a fully online coupled model  system composed of a numerical weather
  forecast model  and a  chemical transport model.   \thanks{Grants or
    other notes  about the  article that should  go on the  front page
    should be placed here.}}
%\subtitle{Write here subtitle}

%\titlerunning{Short form of title} % if too long for running head

\author{J.Charles \and M.Dolz}

%\authorrunning{Short form of author list} % if too long for running
%head

\institute{Dr. Joseph Charles \at Swiss National Supercomputing Centre
  (CSCS)\\ CH-6900  Lugano, Switzerland \\  Tel.: +41 (0) 91  610 8216
  \\  \email{joseph.charles@cscs.ch}  \\  \and  Dr.  Manuel  Dolz  \at
  University of Hamburg \\ 20148  Hamburg, Germany \\ Tel.: +49 (0) 40
  460094-404 \\ \email{manuel.dolz@informatik.uni-hamburg.de}}

\date{Received: date / Accepted: date} % The correct dates will be
                                       % entered by the editor

\maketitle

\begin{abstract}
  In this paper we present  COSMO-ART, an extension of the operational
  weather  forecast  model  of   the  German  Weather  Service  (DWD),
  developed for  the evaluation of the interactions  of reactive gases
  and aerosol particles  with the state of atmosphere  at the regional
  scale. It  includes secondary aerosols,  directly emitted components
  like  soot,  mineral  dust,  sea  salt and  biological  material  as
  pollen. Processes such  as emissions, coagulation, condensation, dry
  deposition,  wet removal,  and sedimentation  of aerosols  are taken
  into account.   The overall performance  of this application  on HPC
  systems is  analysed by a  profiling and tracing study  to determine
  hotspots, i.e.  the  parts of the program that  require the dominant
  fraction of runtime, and to identify critical paths, i.e.  sequences
  of functions that shows dominant inclusive contributions for all its
  elements. Moreover, we describe measurement devices and energy-aware
  techniques  employed  to  evaluate   the  energy  footprint  of  the
  considered  application and  to  get detailed  insights about  power
  bottlenecks.  Our approach is to improve corresponding code sections
  to  sustain high  performance  while minimizing  energy-to-solution.
  This preliminary  study sets the basis of  broader considerations to
  tackle challenges  related to energy efficient HPC  in the framework
  of the Exa2Green project (\url{http://exa2green.eu/}).

\keywords{High performance computing  \and Energy-aware computing \and
  Numerical Weather Prediction}
% \PACS{PACS code1 \and PACS code2 \and more}
% \subclass{MSC code1 \and MSC code2 \and more}
\end{abstract}

\section{Introduction}
\label{intro}
While anthropogenic greenhouse gas emissions are driving unprecedented
major  climate  changes  since   the  mid-20$^{th}$  Century,  one  factor
overwhelmingly  affects  the  uncertainty  in  determining  human-induced
radiative  forcing:  the  effects   of  aerosols  in  the  atmosphere.
Although they  are not considered  heat-trapping greenhouse  gases and
have  shorter  atmospheric lifetimes,  aerosols  significantly modify  the
global   radiation   budget   \citep{IPCC-2013}.   Enhanced   aerosols
concentrations impact the climate  system by scattering and absorption
of solar radiation, thereby exerting a direct radiative forcing; or by
modifying cloud  properties,  cloud  fraction and  surface
albedo,  causing  a  negative  indirect  radiative  forcing.   Despite
considerable  progress in  global aerosol  modelling \citep{Mann-2013}
and  measurement-based  assessments,  large  uncertainties  remain  in
current  estimates  of  aerosol radiative  forcing  \citep{Myhre-2013,
IPCC-2013,  Lee-2013,   Randles-2013,  Rosenfeld-2013,  Sherwood-2013,
Stier-2013}.

Hence to  improve our understanding of  aerosol-cloud interactions and
reduce  these  uncertainties,  the  research  community  is  making  a
concerted international  effort to represent  the underlying physical,
chemical  and aerosol dynamical  processes through  numerical chemical
transport  models (CTMs)  such  as ART  (Aerosols  and Reactive  Trace
gases),   developed   at  the   Karlsruhe   Institute  of   Technology
(KIT)  \citep{Vogel-2009,  Bangert-2011,  Knote-2013}.  This  regional
scale modelling system is coupled  with the operational  weather forecast
model  \textsc{Cosmo}  \citep{Baldauf-2011},  jointly developed and used by  a
consortium of European weather centers, as well as utilised in a climate version
by the wider research community.

The extended  atmospheric model \textsc{Cosmo-art}  is com\-put\-ationally
much  more  demanding than  \textsc{Cosmo}  since  a  large number  of
additional  tracers and  processes have  to be  considered.   Thus the
model  is currently  severely limited  in terms  of applicability  and
expensive in terms of energy consumption.  Although \textsc{Cosmo} has
recently been ported to GPUs \citep{Gysi-2014, Lapillonne-2014} within
the framework of the  High Performance and High Productivity Computing
(HP2C) Initiative  \citep{HP2C} to optimise it  for computational and
energy efficiency,  significant investments in ART  are still required
to take it  to a similar level.  The  efficiency of \textsc{Cosmo-art}
is being  addressed in the  EU Exa2Green project  \citep{EXA2GREEN} to
deliver a  prototype code, which  provides an energy efficiency  of at
least five times the  baseline value.  Such an implementation would
allow  the  community  to  investigate critical  questions  at  higher
resolution  and   over  longer  periods,   at  reduced  cost   to  the
environment.

This  work is  organized as  follows: in  Sec.~\ref{sec:1} we  give an
overview  of  related  work,   then  in  Sec.~\ref{sec:2}  we  briefly
introduce \textsc{Cosmo-art}  and specify its  technical setup related
to  the investigated  performance and  energy evaluation  methods.  In
Sec.~\ref{sec:3}   we  describe   HPC  platforms,   power  measurement
equipments   and  software  environment   employed  to   conduct  this
benchmarking study.  Sec.~\ref{sec:4}  presents performance and energy
requirements  of the  baseline on  these architectures  and highlights
areas where improvements will be necessary for the subsequent baseline
refactoring.   Finally,  we  conclude in Sec.~\ref{concl} with some 
implications for the Exa2Green project and give an outlook for future
research.


\section{COSMO-ART model description}
\label{sec:1}
\subsection{Model description}
\label{subsec:1.1}
COSMO-ART  (ART  stands  for   Aerosols  and  Reactive  Trace  gases),
developed  at  KIT  Karlsruhe  \citep{Vogel-2009}, is  a  regional  to
continental scale model coupled  online to the COSMO numerical weather
prediction  (NWP) and  climate  model \citep{Baldauf-2011}.   Physical
processes  like  transport,  turbulent  diffusion,  and  dry  and  wet
deposition  are  treated  together  with  photochemistry  and  aerosol
dynamics using  the modal  approach.  Aerosols dynamics  are simulated
with the modal aerosol module MADE \citep{Ackermann-1998}, improved by
explicit  treatment of  soot aging  through condensation  of inorganic
\citep{Riemer-2003}   and   organic   substances   (MADEsoot),   which
represents the  aerosol population using  eleven overlapping lognormal
modes.   Five  modes  represent  sub-micron  particles  consisting  of
sulphate,  ammonium,  nitrate,   organic  compounds,  water  and  soot
\citep{Riemer-2004}  in a  range  of mixing  state.   These modes  are
coupled with  the gas  phase by condensation  and nucleation,  and are
strongly influenced by anthropogenic emissions of gases and particles.
Sea    salt     \citep{Lundgren-2013},    mineral    dust    particles
\citep{Vogel-2006, Stanelle-2010}  and rest anthropogenic  species are
treated by  seven additional modes. For each  mode, mass contributions
and total  number concentration  are prognostic quantities,  while the
standard deviation is fixed.

Specific  modules are included  to simulate  the dispersion  of pollen
grains    \citep{Vogel-2008}   and    other    biological   particles.
Meteorologically-influenced emissions  are also online  coupled within
the  model  system.  The  biogenic  VOC  (volatile organic  compounds)
emissions are  calculated as functions of  the land use  type based on
the Global  Land Cover 2000  dataset and the modeled  temperatures and
radiative fluxes \citep{Vogel-1995}.

The gaseous chemistry in COSMO-ART  is solved by a modified version of
the  Regional  Acid  Deposition  Model, Version  2  (RADM2)  mechanism
\citep{Stockwell-1990},  which  is   extended  to  describe  secondary
organic aerosol formation based on a volatile basis set (VBS) approach
\citep{Athanasopoulou-2013}  and  hydroxyl  radical recycling  due  to
isoprene chemistry \citep{Geiger-2003}.  The thermodynamic equilibrium
between the  gas and particulate  phases of the inorganic  material is
achieved through the  ISORROPIA II module \citep{Fountoukis-2007}. 

COSMO-ART  is  fully  online-coupled,  and  allows  for  feedbacks  of
aerosols  on  temperature,  radiation  and cloud  condensation  nuclei
(CCN).   Analytical  description  of  modules  incorporated  for  this
purpose  exists in  \citep{Vogel-2009,  Bangert-2011}.  The  radiation
scheme used  within the  model to calculate  the vertical  profiles of
shortwave and longwave radiative fluxes is GRAALS \citep{Ritter-1992}.
In order to account for  the interaction of aerosol particles with the
cloud microphysics and radiation,  COSMO-ART uses the two-moment cloud
microphysics  scheme of  Seifert and  Beheng  \citep{Seifert-2006} and
comprehensive parameterizations for cloud condensation and ice crystal
nucleation  \citep{Bangert-2011, Bangert-2012}.   The system  of stiff
ordinary differential equations described by chemical reactions in the
aqueous-phase together with the transfer reactions is solved using the
Kinetic PreProcessor (KPP, \citealp{Damian-2002}).

Detailed  model  description  can   be  found  in  the  aforementioned
publications  as   well  as  in   \citep{Stanelle-2010,  Bangert-2012,
  Knote-2011, Knote-2013}.

\subsection{Model setup}
\label{subsec:1.2}
To define a baseline within a code under development, it was necessary
to  find  a  run-configuration  capable  of  being  recreated  in  all
subsequent versions. The  energy-to-solution benchmarking of COSMO-ART
concerns  one-day  long simulations  without  spin-up  stage, using  a
222x216x40  discretization of  Europe and  a  time step  of 120s.  The
baseline code incorporates 34 2-d and  45 3-d fields to be written out
every hour for a simulation starting  on April 13th, which is close to
the equinox and thus brings  benefits of having approximately half day
of night and  half day of sun exposure, to  ensure a proper activation
of  the chemistry  cycle. Ultimately  new  input files  closer to  the
equinox (on March 20th) will be  created, but the current set of input
files is fine for the time being.

This COSMO-ART  version is configured  to deal with a  semi Lagrangian
horizontal advection scheme  with tricubic interpolation and selective
filling diffusion option in combination with the Runge-Kutta dynamical
core.   Concerning the modelling  of wet  deposition in  aerosols, the
baseline  has  only  indirect  cloud  feedbacks  but  doesn't  include
in-cloud  scavenging (rainout)  and  below-cloud scavenging  (washout)
yet.  Amongst  physical parameterizations, precipitation  formation is
performed  by a  two-moment  cloud microphysics  scheme  instead of  a
classical  bulk microphysics  scheme. Another  important point  is the
fact that  this version makes  use of the Kinetic  PreProcessor solver
for  the  resolution of  atmospheric  chemistry ordinary  differential
equations.


\subsection{Subsection title}
\label{subsec:1}
Don't forget to  give each section and subsection  a unique label (see
Sect.~\ref{sec:1}).
\paragraph{Paragraph headings} Use paragraph headings as needed.
\begin{equation}
a^2+b^2=c^2
\end{equation}

% For one-column wide figures use
\begin{figure}
% Use the relevant command to insert your figure file.
% For example, with the graphicx package use
  \includegraphics{example.eps}
% figure caption is below the figure
\caption{Please write your figure caption here}
\label{fig:1}       % Give a unique label
\end{figure}
%
% For two-column wide figures use
\begin{figure*}
% Use the relevant command to insert your figure file.
% For example, with the graphicx package use
  \includegraphics[width=0.75\textwidth]{example.eps}
% figure caption is below the figure
\caption{Please write your figure caption here}
\label{fig:2}       % Give a unique label
\end{figure*}
%
% For tables use
\begin{table}
% table caption is above the table
\caption{Please write your table caption here}
\label{tab:1}       % Give a unique label
% For LaTeX tables use
\begin{tabular}{lll}
\hline\noalign{\smallskip}
first & second & third  \\
\noalign{\smallskip}\hline\noalign{\smallskip}
number & number & number \\
number & number & number \\
\noalign{\smallskip}\hline
\end{tabular}
\end{table}

\begin{acknowledgements}
The  authors  would  like  to  thank the  High  Performance  and  High
Productivity  Computing  (\url{www.hp2c.ch})  Initiative and  the  FP7
funding this project.)
\end{acknowledgements}

%\bibliographystyle{spbasic}  % basic style, author-year citations
%\bibliographystyle{spphys}   % APS-like style for physics
%\bibliographystyle{spmpsci}  % mathematics and physical sciences
\bibliographystyle{unsrtnat}

\bibliography{draft}  % name your BibTeX data base

\end{document}

