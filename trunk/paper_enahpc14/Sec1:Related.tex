While the  TOP500 list was  introduced over 20  years ago to  rank the
performance of HPC systems  worldwide, it is quite recently that energy
efficiency  become  a  critical  constraint  in the  way  to  exascale
computing.   Since 2007,  the Green500  power  measurement methodology
encourages the design,  procurement and management of energy-efficient
infrastructures to contain performance in an affordable and competitive
power  envelope.   However,  the state-of-the-art  research  assessing
performance  and energy  efficiency  of applications  is still scarce.

\emph{Padoin et al.}   \cite{Padoin-2013} investigate performance and
power  consumption  of  an  agroforestry  application  and  show  that
changing workload  can drastically improve energy efficiency  of CPU +
GPU heterogeneous architectures.

\emph{Ou  Pang et  al.}   \cite{Ou-2012} compare  ARM  and Intel  x86
workstations  clusters  and   conclude  that  ARM-based  clusters  are
advantageous  with   lightweight  applications  in   terms  of  energy
efficiency.

\emph{G\"oddeke et al.}  \cite{Goddeke-2013} evaluate weak and strong
scalability of PDE solvers on a cluster of 96 ARM dual-core processors
and demonstrate  that the ARM-based  cluster can be more  efficient in
terms of energy-to-solution compared to x86-based cluster.

\emph{Wittmann  et  al.}   \cite{Wittmann-2013}  perform  a  thorough
analysis of  a lattice-Boltzmann method based CFD  simulation on Intel
Sandy Bridge  processors   and  show   extrapolated  results   on  a
petascale-class machine.

% \emph{Cumming  et  al.}   \cite{Cumming-2014}  present a  simple  and
% practical  methodology  looking at  energy  minimization  that can  be
% applied to various applications.
