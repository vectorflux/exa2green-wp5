While anthropogenic greenhouse gas emissions are driving unprecedented
major  climate changes  since  the mid-20$^{th}$  Century, one  factor
overwhelmingly  affects the  uncertainty in  determining human-induced
radiative  forcing:  the  effects   of  aerosols  in  the  atmosphere.
Although they  are not  considered heat-trapping greenhouse  gases and
have shorter atmospheric  lifetimes, aerosols significantly modify the
global   radiation   budget   \cite{IPCC-2013}.    Enhanced   aerosols
concentrations impact the climate  system by scattering and absorption
of solar radiation, thereby exerting a direct radiative forcing; or by
modifying cloud properties, cloud fraction and surface albedo, causing
a negative indirect  radiative forcing.  Despite considerable progress
in  global aerosol  modelling  \cite{Mann-2013} and  measurement-based
assessments,  large  uncertainties  remain  in  current  estimates  of
aerosol  radiative   forcing  \cite{IPCC-2013,  Lee-2013,  Myhre-2013,
Randles-2013, Rosenfeld-2013, Sherwood-2013, Stier-2013}.

Hence to  improve our understanding of  aerosol-cloud interactions and
reduce  these  uncertainties,  the  research  community  is  making  a
concerted international  effort to represent  the underlying physical,
chemical  and aerosol dynamical  processes through  numerical chemical
transport  models (CTMs)  such  as ART  (Aerosols  and Reactive  Trace
gases),   developed   at  the   Karlsruhe   Institute  of   Technology
(KIT)  \cite{Vogel-2009,  Bangert-2011,  Knote-2013}.   This  regional
scale  modelling  system  is  coupled  with  the  operational  weather
forecast  model \textsc{Cosmo} \cite{Baldauf-2011},  jointly developed
and  used by  a consortium  of European  weather centers,  as  well as
utilised in a climate version by the wider research community.

The extended  atmospheric model \cosmoart  is com\-put\-ationally much
more demanding than \textsc{Cosmo}  since a large number of additional
tracers  and processes  have  to  be considered.   Thus  the model  is
currently severely limited in  terms of applicability and expensive in
terms  of energy  consumption.  Although  \textsc{Cosmo}  has recently
been  ported  to  GPUs  \cite{Gysi-2014, Lapillonne-2014}  within  the
framework  of the  High  Performance and  High Productivity  Computing
(HP2C)  Initiative \cite{HP2C}  to optimise  it for  computational and
energy efficiency,  significant investments in ART  are still required
to take it  to a similar level.  The efficiency  of \cosmoart is being
addressed in  the EU Exa2Green  project \cite{EXA2GREEN} to  deliver a
prototype code, which  provides an energy efficiency of  at least five
times  the baseline  value.  Such  an implementation  would  allow the
community to  investigate critical questions at  higher resolution and
over longer periods, at reduced cost to the environment.

This work is  organised as follows: in Section~\ref{sec:1}  we give an
overview  of  related work,  then  in  Section~\ref{sec:2} we  briefly
introduce  \cosmoart and specify  its technical  setup related  to the
investigated   performance   and   energy  evaluation   methods.    In
Section~\ref{sec:3}  we  describe  HPC  platforms,  power  measurement
equipments   and  software  environment   employed  to   conduct  this
benchmarking  study.   Section~\ref{sec:4}  presents  performance  and
energy  requirements  of  the  baseline  on  these  architectures  and
highlights  areas  where  improvements   will  be  necessary  for  the
subsequent   baseline   refactoring.     Finally,   we   conclude   in
Section~\ref{concl} with  some implications for  the Exa2Green project
and give an outlook for future research.

