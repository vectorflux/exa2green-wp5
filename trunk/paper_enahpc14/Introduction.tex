While anthropogenic greenhouse gas emissions are driving unprecedented
major  climate  changes  since   the  mid  20th  Century,  one  factor
overwhelmingly  affects  the  uncertainty  in  defining  human-induced
radiative  forcing:  the  effects   of  aerosols  in  the  atmosphere.
Although they  are not considered  a heat-trapping greenhouse  gas and
have  shorter  atmospheric lifetimes,  they  significantly modify  the
global   radiation   budget   \citep{IPCC-2013}.   Enhanced   aerosols
concentrations impact the climate  system by scattering and absorption
of   solar   radiation,    thereby   exerting   a   direct   radiative
forcing  \citep{Liao-2005,  Bangert-2012,  Lundgren-2013}; or  by  the
modification of  cloud properties, cloud fraction  and surface albedo,
causing  a negative  indirect  radiative forcing  \citep{Haywood-2000,
Lohmann-2005,     VandenHeever-2011,     Rosenfeld-2013}.      Despite
considerable  progress in  global aerosol  modelling \citep{Mann-2013}
and    measurement-based    assessments   \citep{Myhre-2009},    large
uncertainties  remain  in   current  estimates  of  aerosol  radiative
forcing   \citep{Myhre-2013,    IPCC-2013,   Lee-2013,   Randles-2013,
Rosenfeld-2013, Sherwood-2013, Stier-2013}.

Hence to  improve our understanding of  aerosol-cloud interactions and
reduce  these  uncertainties,  the  research  community  is  making  a
concerted international  effort to represent  the underlying physical,
chemical  and aerosol dynamical  processes through  numerical chemical
transport  models (CTMs)  such  as ART  (Aerosols  and Reactive  Trace
gases),   developed   at  the   Karlsruhe   Institute  of   Technology
(KIT)  \citep{Vogel-2009,  Bangert-2011,  Knote-2013}.  This  regional
scale  model system  is online  coupled with  the  operational weather
forecast  model  COSMO \citep{Baldauf-2011},  jointly  developed by  a
consortium of European weather centers and used in the climate version
(COSMO-CLM) by a wide research community.

The extended  COSMO-ART model  is computationally much  more demanding
than the  COSMO core  since a large  number of additional  tracers and
processes  have  to  be  considered.   Thus this  model  is  currently
severely-limited in  terms of applicability and expensive  in terms of
energy consumption.   Although COSMO has recently been  ported to GPUs
within  the framework of  the High  Performance and  High Productivity
Computing  (HP2C)  Initiative to  optimize  it  for computational  and
energy efficiency (\url{http://www.hp2c.ch/}), significant investments
in  ART  are still  required  to  take it  to  a  similar level.   The
efficiency of COSMO-ART is being addressed in the EU Exa2Green project
(\url{http://exa2green.eu/}). The  ultimate goal of the  project is to
deliver a  prototype code, which  provides an energy efficiency  of at
least five times of the  baseline value.  Such an implementation would
allow  the  community  to  investigate critical  questions  at  higher
resolution  and   over  longer  periods,   at  reduced  cost   to  the
environment.

This work is organized as follows: in Sec.~\ref{sec:1} we describe related work, then in
Sec.~\ref{sec:2}, we briefly  introduce the regional atmospheric model
COSMO-ART and specify its  technical setup related to the investigated
performance and energy evaluation methods.  Sec.~\ref{sec:5} describes
the HPC supercomputers used to compute the baseline. Sec.~\ref{sec:3} describes
different  power  measurement  equipment  used  for  conducting  this
benchmarking  study and presents  energy-aware techniques  employed to
determine  the   energy  footprint  of   the  considered  application.
Sec.~\ref{sec:4} gathers benchmark  results from the Exa2Green project
and  highlights areas  where improvements  will be  necessary  for the
subsequent baseline refactoring.  Finally, we conclude with practical
guidelines  to identifying  energy  saving opportunities  and give  an
outlook to future research in Sec.~\ref{concl}.
